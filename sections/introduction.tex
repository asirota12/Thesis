To understand the details of this thesis, a few questions have to be answered first. What is a topological insulator? What is a fractional topological insulator? What are Dirac and Weyl semimetals? What are quasiparticles? What do many body interactions change in these settings? In this section the necessary background will be given to understand both the questions addressed and many of the techniques used to solve them. This will support the two papers this thesis is based upon ~\cite{SahooSirotaChoTeo17,RazaSirotaTeo17}.

\subsection{What is the quantum Hall effect?}

A first logical step is to go over the classical Hall effect. A simple model is the 2D electron gas. When an electric field is applied, certain materials produce a current in the perpendicular direction. An important quality here is that such a current breaks time reversal symmetry. There are two directions the current could choose, and yet it only goes in one. An easy way to break this symmetry is by turning on a perpendicular magnetic field. Using a classical model, simple F=ma yields a steady state. If $E=|E|\hat{x}$ and $B=|B|\hat{z}$ then
\begin{align}
F = -e(E+v \times B) = 0 \\
|E|\hat{x} + v \times |B| \hat{z} =0 \\
v = -|E|/|B| \hat{y}\\
j = -nev = ne|E|/|B| \hat{y}
\end{align}
where n is the density of electrons. Now, $J= \sigma E$, where $\sigma$ is the conductance tensor. In this case it is a 2x2 matrix. Since the current is in the $\hat{y}$ direction and the electric field is in the $\hat{x}$ direction, this gives $\sigma_{xy}= ne/|B| $ and $\sigma_{xx}= 0 $. Assuming the material is rotationally invariant, $\sigma_{yy}= 0 $ and $\sigma_{yx}= -ne/|B| $. The resistivity tensor is just the inverse of the conductance tensor.
\begin{align}
\sigma = 
\begin{pmatrix}
0 & ne/|B| \\
-ne/|B| & 0
\end{pmatrix} \quad 
\rho = 
\begin{pmatrix}
0 & |B|/(ne) \\
-|B|/(ne) & 0
\end{pmatrix}
\end{align}
Experimental results confirm this for low B fields, but gives us quite stunning effects in larger fields, as seen in Figure \ref{fig:quantumhall}. This is not linear at all in B! By defining the filling factor $\nu=nh/eB$, $\sigma_{xy}$ can be rewritten as $ = \nu e^2/h$. It appears that this $\nu$ is constant and integer valued on these flat sections of the data, which we call quantum Hall plateaus. 

\begin{figure}
	\centering
	\includegraphics[width=0.5\linewidth]{images/QuantumHall}
	\caption{von Klitzing won the Nobel Prize in 1985 for discovering the quantum Hall effect evidenced by the above platueas. The green lines show $\rho_{xx}$ and the red $\rho_{xy}$. }
		
		%from http://www.referatele.com/referate/fizica/online4/HALL-EFFECT---Explanation-of-the-Quantum-Hall-Effect-referatele-com.php$
	
	\label{fig:quantumhall}
\end{figure}

Laughlin has a good charge pump argument for this quantization. A material exhibiting this behavior with $\nu = $ some integer at some specific magnetic field could be placed on a cylinder of radius R with periodic boundary conditions. Using the coordinate x = x + R for the azimuthal direction and y between 0 to L for the vertical direction, the allowed momentum around the cylinder is quantized as $k_x= 2\pi n/R$ for n an integer. Then if an extra flux $\Phi$ goes through the cylinder, the vector potential can be chosen as $A(\theta,y) = (By+\Phi /R)\hat{x}$. Of note is that the $By$ term gives a magnetic field B in the z direction, which in this case is the radial direction. This is the magnetic field through the material. The constant $\Phi/R$ just gives increases the flux through the cylinder by $\Phi$. This gives the Hamiltonian 
\begin{align}
H = \frac{p_y^2 +(p_x-eA)^2}{2m} = \frac{p_y^2 +(\hbar k_x - eA)^2}{2m} \\
= \frac{p_y^2}{2m} +\frac{(\hbar 2\pi n/R-eBy-e\Phi /R )^2}{2m}\\
= \frac{p_y^2}{2m} +\frac{(eB)^2}{2m}(y-\frac{hn}{eBR}+\frac{\Phi }{RB} )^2
\end{align}
Which is easily recognized as the simple harmonic oscillator Hamiltonian, just with y shifted by $\frac{hn}{eBR}+\frac{\Phi }{RB}$ or $(n-\frac{e\Phi}{h})\frac{h}{eBR}$, with $\frac{1}{2}m\omega^2 = \frac{(eB)^2}{2m}$
Notice that if $\Phi$ is shifted by $h/e$, that is the same as replacing n with n+1 in the Hamiltonian. This is what defines the flux quantum $\Phi_0$. So the spectrum is a series of states that are free in the x direction, and harmonic oscillators in the y direction, each one centered at some y that depends on the momentum in the x direction. If a flux quantum is added through the cylinder, that moves all the quantum states vertically by one spacing; it acts like an incompressible liquid. This essentially moves charge from one end of the cylinder to the other end. Since there is an integer number of filled harmonic oscillator states on each oscillator center, and the Hamiltonian goes back to itself after increases $\Phi$ by $\Phi_0$, this means an integer number of electrons has been moved. Now $\Delta Q = I \Delta T = \sigma_{xy}E \Delta T = \sigma_{xy} \Delta \Phi = \sigma_{xy} h/e = \nu \frac{e^2h}{he} = \nu e $ . This argument shows that $\nu$ is an integer and thus explains the integer quantum Hall effect! If one of the edges of the cylinder is shrunken to a point, then that is the point where the flux goes through. If charge is pumped to it by changing the flux, there will be some electrons bound to the flux at a general point in a 2D material. Since it is localized, it is in some sense a composite particle, and may have interesting properties.

\subsection{What is the fractional quantum Hall effect?}

The extrapolation of this to the fractional quantum Hall effect is actually quite simple. In the case where $\nu$ is a fraction, with the same set up as the previous charge pump problem where the flux through the cylinder is changed from $\Phi$ to $\Phi + \Phi_0$, the same results are derived. The difference is a charge of $\nu e$ has been moved from one side to another which is now not an integer number of electrons. Since this has been done adiabatically, this is an eigenstate. This implies that there exist local excitations that have a fractional amount of charge. Of course, these states are built out of electrons, and are just parts of a many-body electron wave functions. Again when one edge of the cylinder is shrunken to a point, then the state is tied to a flux. Then there is a fully movable local quasiparticle with fractional charge and flux. Another important consequence is described through particle exchange. Normally if two fermions are exchanged, the wavefunction picks up a sign, or a phase of $\pi$. If they were two bosons the phase is 0. Here there is a fractional charge though, so there is no reason for that to hold. If these two excitations are braided, or exchanged twice, or topologically equivalent to having one particle make a circle around the other and return to its original position, there can be some phase when compared to moving around the same circle without the second particle in the center. It is called braiding since the world line of these particles looks exactly like a braid. In 3 dimensions or higher, if a particle is braided around another, that is topologically equivalent to doing nothing, but in 2 dimensions, in general there are an integer number of classes of distinct topological paths, corresponding to how many times on particle wraps around the other. This is why you can get complex braiding phases. Particle statistics are in a sense, a representation of the braid group acting on the wavefunction. Back to our case, if there is one stationary quasiparticle tied to a quantum of flux, and then another particle of charge $e\nu$ moved around it, then there will be an phase of $e\nu\Phi_0/\hbar$ or $2\pi\nu$. That means the exchange phase has to be $\pi \nu$ which if $\nu$ is a fraction is completely new. The topological spin $h$ is defined in terms of these braiding phases, with $\theta_{a}=e^{2\pi i h}$, where $\theta_{a}$ is the exchange phase of a with a. As a sanity check, a fermion with spin 1/2 has a -1 phase and bosons of spin 0 have a phase of 1. This is why these quasiparticles are called anyons, as opposed to fermions and bosons, since they can in principle have "any" spin. 

Laughlin described a trial wavefunction that approximates well the $\nu = 1/q$ ground state, which gives $> 99\%$ overlap with the numerically calculated ground state.

Laughlin's wave function is 
\begin{align}
\Psi(z_1,z_2,...,z_n) = \prod_{j<k}(z_j-z_k)^q \exp^{-\sum_i(|z_i|^2eB/4\hbar)}
\end{align} 
The form can be guessed by assuming it has a "Jastrow" form component $\prod_{j<k}f(z_j-z_k)$, which is translation invariant, and takes into account two body interactions. The exponential term localizes each electron. Notice that q has to be odd in order for the wavefunction to be antisymmetric. Laughlin also described an operator which locally creates the excitation at $z_0$ with -1/q charge, or removes a flux quanta, which is just $\prod_{j<k}(z_j-z_0)$. This can be used to find out the braiding statistics, and the charges. It can be easily guessed however that since an electron adds a  $\prod_{j<k}(z_j-z_0)^q$, along with an exponential suppression, that this operator is just 1/q of that. There is experimental evidence of these fractional charges\cite{1997Natur.389..162D}. Haldane and later Halperin showed that if there is a Laughlin state, some of the excitations can be used to create another Laughlin state; much like how the Laughlin state is made of electron operators. This yeilds filling fractions of any odd denominator. These are called hierarchy states. 

There is a basic picture to understand what is happening here described by Jain. $\nu = \frac{nh}{eB}$, but $\frac{h/e}{BA}$ where A is the area of the state, is one over the number of flux quantums, and n is the density of electrons. That means $\nu = \frac{nhA}{eBA} = \frac{nA}{\#\Phi_0} = \frac{\#elec}{\#\Phi_0}$. So the fraction really describes a ratio between fraction of electrons to flux quanta. The idea is to make a non interacting composite system. Electrons don't interact with even amounts of fluxes, since they get $\pi$ phases around one, or $2\pi$ with 2. To recreate our original state out of just composite fermions and a magnetic field, there must be the same filling fraction $\nu$, or $\nu$ electrons for each flux for these composites. So the density $\nu$ per flux of composites is necessary to get the correct charge, but a density of $2m\nu$ of fluxes on top of the external flux is necessary to cancel out the even number of fluxes attached to the composites. So then each composite sees a magnetic flux density of $2m\nu+1$. An integer quantum Hall state with filling $p = \frac{\nu}{2m\nu+1}$ can be created out of composites, which corresponds to an original state with a filling fraction for electrons of $\nu = \frac{p}{2mp-1}$. The integer quantum Hall state is already a non compressible non interacting liquid, so this functions as a model for the fractional state. 

Another interesting fact about fractional quantum Hall states is when they are placed on a torus, there is ground state degeneracy. If a flux and an antiflux are introduced at some point, they can go around one of the nontrivial torus loops and annihilate. Now label the 2 different ways you can do this by $T_1$ and $T_2$. It is possible to act with the operators in the following way, $T_1^{-1}T_2^{-1}T_1T_2$, which is the commutator of these 2 operators. Now after a T operator has acted, there is left behind a flux line that loops around the torus. In general these fluxes can braid, which means their commutator is in general non-zero. Yet, these T operators certainly commute with the Hamiltonian. This means if the state starts in a ground state that is also an eigenstate of $T_1$, $|\psi>$, and goes to $T_2|\psi>=e^{i\theta}|\psi>$ under $T_2$, that would mean the T's commute. However they don't so $T_2$ must bring the system to a distinct ground state, i.e. topological ground state degeneracy. In general the braiding phases mentioned earlier, don't just have to be a phase, since they act on now a vector space of ground states, i.e. it could be a represented as a matrix on the space of ground states. This is known as a non Abelian topological phase.

The formulation of two very special things happened in this section; quasiparticles with fractional charge and with fractional statistics. Regarding the problem of classifying fractional quantum Hall states, one could look at either symmetries or topological invariants. If two states have topologically distinct quasiparticles, then they are definitely different phases, since the braiding rules would not change continuously. Even if the restriction to the class without symmetry in just 2 dimensions is considered, there could possibly be a fantastic number topologically distinct quasiparticles, thrown into an even larger size of different states. It turns out there are rules that govern what types of quasi-particles you can have, like locality and unitarity, that place this into the branch of mathematics called modular tensor category theory. Even then this is not a solved problem. This is very abstract still, since only a few of the many possible states allowed by modular tensor category theory have been experimentally realized, and many of those states can be well understood using the composite description above.

\subsection{The Pfaffian states}

So of interest in this work is not an odd denominator state, but even. It would be useful to be able to have an operator that is like half of the electron creation operator. The reason being that for the Dirac semimetal paper~\cite{RazaSirotaTeo17}, the Dirac electrons could be written as $v=1/2$ operators, and then possibly interactions with these operators can gap the semimetal in a symmetric way that the original electron operators could not. That would be inherently a many body phase, since these operators must be built of parts of manybody electron wavefunctions.  This is precisely the solution this paper is looking for.

The Moore Read state~\cite{MooreRead,ReadMoore,GreiterWenWilczekPRL91,GreiterWenWilczek91} provides a possible description of a $v=5/2$ state. The first two Landau levels are inert, so just the half filled third level is considered. Notably these states then have $v=1/2$ theoretically, even though there is no $v=1/2$ state experimentally. The two "inert" Landau levels are actually vital to the states existence, but the resulting ground state of this approximation seem to match well with the $v=5/2$ state. Essentially there is the same composite fermion description, i.e. two fluxes for each electron. Now instead of making an integer quantum Hall state with them, which would only yield odd denominator filling fractions, just like in superconducting theory, they combine to make a "Cooper pair" and then open up a "superconducting" gap, or a nonzero energy cost to create excitations. This can create a gap in a number of ways, which gives rise to many possible states. Since these are not actual electrons it does not mean the actual state is superconducting. In particular Moore and Read wrote down a trial wave function very similar to the Laughlin wavefunction.

In order to know how they got this wavefunction, some conformal field theory is necessary. Conformal field theory will be useful later, so there will be a brief overview here.

\subsubsection{Conformal Field Theory and Chern-Simons Theory Aside}
Conformal field theory evolved from string theory. It describes field theories that have a conformal symmetry, which is a spatial/time transformation that leaves the metric $g_{\mu \nu}$ unchanged up to multiplying it by a constant. It turns out that in 1+1 dimensions, at a critical point, if there is a local Lagrangian there is a conformal symmetry. Even for non critical theories such as fractional quantum Hall theories, if they are Abelian they are described by a Chern-Simons theory. Witten showed the Chern-Simons 2+1 d excitations correspond to primary fields on the edge. Abelian Chern-Simons theories describe most of fractional quantum Hall states, and there is powerful formalism for them. The Abelian Chern-Simons term is a gauge invariant term in 2+1 dimensions, 
\begin{align}
\mathcal{L}_{CS} = \frac{1}{4\pi}\varepsilon^{\mu \nu \lambda }K_{IJ}a_\mu^I\partial_\nu a^J_\lambda -e A_\mu t_I \partial_\nu a^I_\lambda \varepsilon^{\mu \nu \lambda }
\end{align}
Which corresponds to 1+1 d conformal field theory of 
\begin{align}
\mathcal{L}_{CFT} = \frac{1}{2\pi}\partial_t\phi^I K_{IJ} \partial_x\phi^J + ...
\end{align}

Where the NxN coupling matrix K is creatively called the "K matrix", $a$ and $\phi$ are N component U(1) gauge fields, A is the electromagnetic gauge field and the coupling t is a called the charge vector. K must be integer valued to be gauge invariant. The low energy excitations of this theory are given by the potential energy, meaning not just any combinations of $\phi$'s are there. This matrix follows through to the conformal field theory, and together with the charge vector, can be used to calculate what all the excitations are in terms of the fields, their fusion rules and their braiding statistics. The K matrix defines an integer anyon lattice $\Gamma^*=\mathbb{Z}^N$, where a vector $b=(b_1,b_2,...b_N)$ corresponds to a creation operator of an anyon $\psi_b=e^{i b \cdot \phi}$ on the ground state. Fusion then is just vector addition in this lattice. The charge of this anyon is given by $q_a = t^TK^{-1}a$. Many of these particles are local bosons, specifically, any excitations given by $K\mathbb{Z}^N$ do not braid with anything, so anyons are typically considered mod any local boson, or just anyons in $\mathbb{Z}^N/K\mathbb{Z}^N$, which contains det|K| elements. The canonical commutation relations of the Lagrangian give the commutation relations for the creation and annihilation operators for these fields. This gives the exchange phases and hence braiding. The braiding phase of anyons a around b is $\theta_{a,b} = e^{2\pi i a^T K^{-1} b}$, and the topological spin of a particle is defined by a 360 twist, or equivalently the phase it is exchanged with itself, i.e., half a braiding phase. This goes as follows: $\theta_a=\sqrt{\theta_{a,a}} = e^{\pi i a^T K^{-1} a} = e^{2 \pi i h_a}$ or $h_a = a^T K^{-1} a /2$ mod 1. It can be checked that an exchange phase of -1 corresponds to spin 1/2. Notice also that braiding and fusion are related. If a 360 twist is done on particle c, and c = a x b, this is also braiding a around b, with a 360 twist of a and b. The $2\pi$ monodromy phase $\mathcal{M}^{XY}_Z=R^{XY}_ZR^{YX}_Z$ between primary fields $X$ and $Y$ with a fixed overall fusion channel $Z$ can be deduced by the {\em ribbon identity}~\cite{Kitaev06}: \begin{align}e^{2\pi ih_Z}=\vcenter{\hbox{\includegraphics[width=0.5in]{ribbon1.jpg}}}=\vcenter{\hbox{\includegraphics[width=0.5in]{ribbon2.jpg}}}=\mathcal{M}^{XY}_Ze^{2\pi i(h_X+h_Y)}\label{ribbonapp}\end{align} for $h_{X,Y,Z}$ being the conformal spins for primary fields $X,Y,Z$. Unlike the gauge-dependent $\pi$-exchange phase $R^{XY}_Z$, the $2\pi$-monodromy phase $\mathcal{M}^{XY}_Z=e^{2\pi i(h_Z-h_X-h_Y)}$ is gauge independent and physical. This means with just the spins and fusion rules the braiding statistics can be derived. Not only that but $\sigma_{xy}$ can be calculated as $tK^{-1}t$  in units of the quantum conductance. This is consistent with the Laughlin theory having a K matrix of 3, and a charge vector of 1. Here is a summary of these important formulas.

\begin{align}
\theta_{a,b} = e^{2\pi i a^T K^{-1} b} \label{eq:transform1}\\
h_a = a^T K^{-1} a /2 \mod 1 \label{eq:transform2}\\
\theta_{a,b}= h_{a\times b}-h_a-h_b  \label{eq:transform3}\\
q_a = t^TK^{-1}a \label{eq:transform4}\\
\sigma_{xy} = t^TK^{-1}t \label{eq:transform5}
\end{align}

A change of basis gives new matrices described by the following 
\begin{align}
\tilde{\phi}=M\phi \quad M^T\tilde{K}M=K \quad \tilde{t}=Mt
\end{align}

Notably, not all theories are Abelian. There are non-Abelian Chern-Simons theories, they are not so straightforward to describe. Here is a description of the general structure of conformal field theory.  For a more in depth introduction, see {\color{red}cite this} http://www.damtp.cam.ac.uk/ user/tong/string/four.pdf. 

In two dimensions, conformal field theory typically uses $z = t + ix$  and $\bar{z}= t - ix$ instead of using space and time coordinates because conformal transformations then take the form $z \rightarrow f(z)$ and $\bar{z} \rightarrow \bar{f}(\bar{z})$. In this way the theory splits into just the z or $\bar{z}$ dependent pieces. The infinitesimal form of this is $z \rightarrow z + \epsilon(z)_n$ where $\epsilon(z)_n = -z^{n+1}$, which is generated by $l_n = -z^{n+1}\partial_z$. $l_0+\bar{l}_0$ and $i(l_0-\bar{l}_0)$ generate scaling and rotations. In a quantum theory, infinitesimal transformations can be generated using the stress energy tensor, and there may be some nontrivial commutation. In conformal theories, the stress energy tensor also breaks up into two parts, $T_{zz}(z)$ and $T_{\bar{z}\bar{z}}(\bar{z})$, with $T_{zz} =T_{\bar{z}\bar{z}}=0$. If the stress energy tensor is broken into moments, $L_n$ they will follow similar commutation relations as $l_n$, except there is an extra piece called the central charge on the commutation of $L_n$ and $L_{-n}$. Their algebra is called the Virasoro algebra.

There is a notion of primary fields, which transform a specific way under a conformal transformations, $ \Psi(z,\bar{z}) \rightarrow (\frac{\partial f}{\partial z})^h (\frac{\partial \bar{f}}{\partial \bar{z}})^{\bar{h}}  \Psi(f(z),\bar{f}(\bar{z}))$, where $h$ and $\bar{h}$ are called the conformal weights. These are important to remember since they are eigenstates of scaling and rotations. $h-\bar{h}$ is the spin eigenvalue and $h+\bar{h}$ is the eigenvalue of scaling. That also makes them eigenstates of Virasoro operators $L_n$, or a representation of the algebra. This is a little different then typical field theory, where we call the functions we integrate the path integral over fields, here a field is just any function.

The next important piece is the operator product expansion. The correlator of two local operators, $<O_1(z,\bar{z})O_2(w,\bar{w})>$ can be expressed by taking a Taylor expansion as w approaches z. The operator product expansion is defined as
\begin{align}
<O_i(z,\bar{z})O_j(w,\bar{w})...> = \sum_k C^k_{ij}(z-w,\bar{z}- \bar{w}) <O_k(w,\bar{w})...>
\end{align}
Where the sum is over all local operators. The ... part refers to the idea that this holds when these are followed by a string of other operators, as long as they are not close to z or w relative to the distance between them. The form of these C functions are restricted since there is conformal symmetry. For example they must only depend on position differences due to translation symmetry. If the local excitations in the quantum Hall effect are the local operators, then their operator product expansion would give information about their fusion rules, i.e., what other local operators their sum is described by. Something familiar to this are the Clebsch–Gordan coefficients. This looks much like an algebra, but because of its z dependence is given the name vertex algebra. It actually contains more information than fusion rules. If the operators are primary they are restricted to only have two divergent parts, with one proportional to the conformal weight. The operator product expansion can be used to define primaries as well.

The operator product expansion of the stress energy tensor with itself, which is notably not primary, will get not just a weight term, but a $1/(z-w)^4$ term proportional to what is called the central charge. The local operator with this piece is just the identity, so it adds some zero energy to your theory. That energy is also known as a Casimir energy. There is a theorem that says that the central change counts your degrees of freedom, and is related to the total heat current. For example $c=\bar{c}=n$ for n scalar fields (bosons), and $c=\bar{c}=n/2$ for n free fermion fields.

Finally there are rational conformal field theories. These are theories where you can get all of the possible fields from a finite number of primary fields. The generators are representations of the Virasoro algebra. By acting on them with "lowering" operators, their conformal weights can be changed. The rational conformal field theories are completely solvable given just symmetry arguments. A good analogy here are the spin states, where given a highest spin, by acting with lowering operators all possible states can be described. At some point the sequence will terminate.

Moore and Read found a way to write the Laughlin wavefuntion as a correlator of fields. Then they used that same prescription to find a wave function for a $\nu=1/2$. The real wavefunction would be this on top of the wavefunction of two completely filled Landau levels. 
\begin{align}
\Psi(z_1,z_2,...,z_n) = Pf(\frac{1}{z_i-z_j})\prod_{j<k}(z_j-z_k)^2 \exp^{-\sum_i(|z_i|^2eB/4\hbar)}
\end{align}

Where $Pf$ is the Pfaffian of a skew-symmetric matrix. It is defined as a polynomial in the matrix entries with integer-valued coefficients such that $Pf(M)^2=det(M)$. The term falls out of Wicks theorem when it is applied to real fermion fields. The important thing about this term is that it lets the wavefunction be antisymmetric when q is even. Using this ground state, the elementary excitations can be found. Since the electrons or composite fermions are paired, when a flux is braided around them the resulting phase is doubled. This means the flux quantum is halved. So now in the charge pump argument the elementary excitation has charge $\nu e/2 = e/4$, modulo e. The braiding statistic of this particle with the electron is 1, and its spin $h$, which is defined by the phase of braiding it around another copy of itself, $e^{2\pi i h}$, is $1/16$. This primary field also ends up being non-Abelian, i.e., the operator product expansion with itself gives a two primary fields, a fermion and a boson with charge e/2. In this way all the excitations are produced. This Moore-Read state can be described as a decomposition into $Ising \otimes U(1)_4$ Where the $U(1)_4$ is just particles of charge $ne/4$, where n is an integer from 0 to 7, called $e_1,e_2,...e_7$, and the Ising theory is the Majorana fermion $\psi$, and the $\pi$ flux $\sigma$, where anything that is not local with respect to the electron is removed. The excitations topological spins are described in Table ~\ref{tab:Pfaff}.
\begin{table}[h]
	\centering
	\begin{tabular}{|c|c|c|c|c|c|c|c|c|}
		\hline
		&$e_0$& $e_1$ & $e_2$ & $e_3$ & $e_4$& $e_5$& $e_6$& $e_7$\\
		\hline
		$\openone$ & 0 && 1/4 & & 0 && 3/4 & \\
		\hline
		$\sigma $ && 1/8 && 5/8 && 5/8 && 1/8 \\
		\hline
		$\psi$ & 1/2 && 3/4 & & 1/2 && 1/4 &\\
		\hline
		charge & 0 &e/4& 2e/4 & 3e/4 & e & 5e/4& 6e/4 & 7e/4\\
		\hline
	\end{tabular}
	\label{tab:Pfaff}
	\caption{The topological spins of the Pfaffian state}
\end{table}

Much of what follows now until the end of this section is a direct exert from one of the papers this thesis is based upon~\cite{RazaSirotaTeo17} that make some important clarification valuable at this point in the discussion. It is important to clarify and disambiguate the three "Pfaffian" fractional quantum Hall states that commonly appear in the literature. All these $(2+1)$D states are theorized at filling fraction $\nu=1/2$, although are applied to $\nu=5/2$ in materials, and have identical electric transport properties. However, they have distinct thermal Hall transport behaviors. They all have very similar anyonic quasiparticle structures. For instance, they all have four Abelian and two non-Abelian quasiparticles (up to the electron). On the other hand, the charge $e/4$ non-Abelian Ising anyons of the three states have different spin-exchange statistics. First, the gapless boundary of the Moore-Read Pfaffian fractional quantum Hall state can be described by the $(1+1)$D chiral conformal field theory $U(1)_4\otimes\mathrm{Ising}$ where the charged boson and neutral fermion sectors are co-propagating. It therefore carries the chiral central charge $c=1+1/2=3/2$, which dictates the thermal Hall response \eqref{conductance}. Second, the "anti-Pfaffian" fractional quantum Hall state~\cite{LevinHalperinRosenow07,LeeRyuNayakFisher07} is the particle-hole conjugate of the Moore-Read Pfaffian state. Instead of half-filling the lowest Landau level by electrons, one can begin with the completely filled lowest Landau level, and half-fill it with holes. In a sense the anti-Pfaffian state is obtained by subtracting the completely filled lowest Landau level by a Moore-Read Pfaffian state. Along the boundary, the $(1+1)$D conformal field theory $U(1)_{1/2}\otimes\overline{U(1)_4\otimes\mathrm{Ising}}$ consists of the forward propagating chiral Dirac $U(1)_{1/2}$ sector that corresponds to the lowest Landau level, and the backward propagating Moore-Read Pfaffian $\overline{U(1)_4\otimes\mathrm{Ising}}$. Here $\overline{\mathcal{C}}$ can be interpreted as the time-reversal conjugate of the chiral conformal field theory $\mathcal{C}$. The thermal transport is governed by the edge chiral central charge $c=1-3/2=-1/2$, which has an opposite sign from the filling fraction. Thus, unlike the Moore-Read Pfaffian state, the net electric and thermal currents now travel in opposite directions along the edge. Lastly, the recently proposed particle-hole symmetric Pfaffian state~\cite{Son15,BarkeshliMulliganFisher15,WangSenthil16}, which is going to be the {\em only} Pfaffian fractional quantum Hall state considered here (see Ref.~\onlinecite{KaneSternHalperin17} for a coupled wire construction), has the chiral edge conformal field theory \eqref{PfaffianCFT}. As the electrically charged boson and neutral fermion sectors are counter-propagating, the net thermal edge transport is governed by the chiral central charge $c=1-1/2=1/2$. The chiral $(1+1)$D particle hole symmetric Pfaffian conformal field theory \eqref{PfaffianCFT} is also present along the line interface separating a time reversal symmetric $\mathcal{T}$-Pfaffian~\cite{ChenFidkowskiVishwanath14} domain and a time reversal breaking magnetic domain on the surface of a 3D topological insulator. (Similar constructions can be applied to alternative time reversal symmetric topological insulator surface states~\cite{WangPotterSenthilgapTI13,MetlitskiKaneFisher13b,BondersonNayakQi13}, but they will not be considered here.) Other than their thermal transport properties, the three Pfaffian fractional quantum Hall state can also be distinguished by the charge $e/4$ Ising anyon, which has spin $h=1/8$, $-1/8$ or $0$ for the Moore-Read Pfaffian, anti-Pfaffian or particle hole symmetry Pfaffian states respectively. 

Since this thesis will not be considering the Moore-Read Pfaffian or its particle-hole conjugate anti-Pfaffian state, the particle hole symmetry Pfaffian state will be refered to simply as the Pfaffian state. It is actually true that there are other Abelian theories that describe a $\nu=1/2$ state, but this thesis does not work with them.

\subsection{Pfaffian Field Theory}
The reason why there is such focus on this Pfaffian state is that many-body interactions can facilitate the fractionalization of a $(1+1)$D chiral Dirac channel \begin{align}\mathrm{Dirac}=\mathrm{Pfaffian}\otimes\mathrm{Pfaffian}\label{fractionalization}\end{align} (see also figure~\ref{fig:glueingsplitting}). In a sense, each chiral Pfaffian channel carries half of the degrees of freedom of the Dirac. For instance, it has half the electric and thermal conductances, which are characterized by the filling fraction $\nu=1/2$ and the chiral central charge $c=1/2$ in \eqref{conductance}. Through out this theis there are references to the low-energy effective theory that consists of an electrically charged $U(1)_4$ bosonic component,conformal field theory say moving in the $R$ direction, and a neutral Majorana fermion component moving in the opposite $L$ direction -- simply as a Pfaffian conformal field theory \begin{align}\mathrm{Pfaffian}=U(1)_4\otimes\overline{\mathrm{Ising}}.\label{PfaffianCFT}\end{align} This thesis follows the level convention for $U(1)$ in the conformal field theory community~\cite{bigyellowbook}. The same theory may be more commonly referred to as $U(1)_8$ in the fractional quantum Hall community. For clarification, see Lagrangian \eqref{Pfaffian} and \eqref{LFQHCS}.)

The low-energy effective chiral $(1+1)$D conformal field theory takes the decoupled form between the boson and fermion \begin{align}\mathcal{L}_{\mathrm{Pfaffian}}&=\mathcal{L}_{\mathrm{charged}}+\mathcal{L}_{\mathrm{neutral}}\label{Pfaffian}\\&=\frac{8}{2\pi}\partial_t\phi_R\partial_x\phi_R+v(\partial_x\phi_R)^2\nonumber\\&\;\;\;+i\gamma_L(\partial_t-\tilde{v}\partial_x)\gamma_L\nonumber\end{align} where $\hbar$ has been set to 1. Here $\phi_R$ is the free chiral $U(1)_4$ boson. It generates the $(1+1)$D theory $\mathcal{L}_{\mathrm{charged}}$, which is identical to the boundary edge theory of the $(2+1)$D bosonic Laughlin $\nu=1/8$ fractional quantum Hall state described by the topological Chern-Simons theory~\cite{WenZee92,Wenedgereview} \begin{align}\mathcal{L}_{2+1}=\frac{K}{4\pi}\alpha\wedge d\alpha+et\alpha\wedge dA\label{LFQHCS}\end{align} with $K=8$ and $t=2$. The $U(1)_4$ conformal field theory carries the electric conductance $\sigma=tK^{-1}t=1/2$ in units of $2\pi e^2=e^2/h$ and a thermal conductance characterized by the chiral central charge $c=c_R=1$. Primary fields are of the form of (normal ordered) chiral vertex operators $:e^{im\phi_R}:$, for $m$ an integer, and carries charge $q=m/4$ in units of $e$ and conformal scaling dimension (i.e.~conformal spin) $h=h_R=m^2/16$. Here is a summary and abbreviation the operator product expansion \begin{align}e^{im_1\phi_R(z)}e^{im_2\phi_R(w)}=e^{i(m_1+m_2)\phi_R(w)}(z-w)^{m_1m_2/8}+\ldots\end{align} by the Abelian fusion rule \begin{align}e^{im_1\phi_R}\times e^{im_2\phi_R}=e^{i(m_1+m_2)\phi_R},\end{align} where $z\sim\tau+ix$ is the complex space-time parameter and $\tau=i\pi vt/2$ is the Euclidean time.

$\gamma_L^\dagger=\gamma_L$ is the free Majorana fermion. It generates the $(1+1)$D theory $\mathcal{L}_{\mathrm{neutral}}$, which is equivalent to a chiral component of the critical Ising conformal field theory or the boundary edge theory of the $(2+1)$D Kitaev honeycomb model~\cite{Kitaev06} in its B-phase with time reversal breaking (i.e.~a chiral $p_x+ip_y$ superconductor coupled with a $\mathbb{Z}_2$ gauge theory). It carries trivial electric conductance but contributes to a finite thermal conductance characterized by the chiral central charge $c=-c_L=-1/2$. The Ising conformal field theory has primary fields $1$, $\gamma_L$ and $\sigma_L$, where the twist field (or Ising anyon) $\sigma_L$ carries the conformal spin $h=-h_L=-1/16$. Again, the operator product expansions go as \begin{gather}\gamma_L(\bar{z})\gamma_L(\bar{w})=\frac{1}{\bar{z}-\bar{w}}+\ldots\nonumber\\\sigma_L(\bar{z})\gamma_L(\bar{w})=\frac{\sigma_L(\bar{w})}{(\bar{z}-\bar{w})^{1/2}}+\ldots\nonumber\\\sigma_L(\bar{z})\sigma_L(\bar{w})=\frac{1}{(\bar{z}-\bar{w})^{1/8}}+(\bar{z}-\bar{w})^{3/8}\gamma_L(\bar{w})\nonumber\end{gather} by the fusion rule \begin{gather}\gamma_L\times\gamma_L=1,\quad\sigma_L\times\gamma_L=\sigma_L\nonumber\\\sigma_L\times\sigma_L=1+\gamma_L,\end{gather} where $\bar{z}\sim\tau-ix$ is the complex space-time parameter and $\tau=i\tilde{v}t$ is the Euclidean time. 

General primary fields of the Pfaffian conformal field theory decompose into the $U(1)_4$ part and the Ising part. They take the form \begin{align}1_m=e^{im\phi_R},\quad\psi_m=e^{im\phi_R}\gamma_L,\quad\sigma_m=e^{im\phi_R}\sigma_L.\label{Pfaffianfields}\end{align} The conformal spins and fusion rules also decompose so that \begin{align}h_{1_m}=\frac{m^2}{16},\quad h_{\psi_m}=\frac{m^2}{16}+\frac{1}{2},\quad h_{\sigma_m}=\frac{m^2-1}{16}\end{align} modulo 1, $q_m=m/4$ in units of $e$, and \begin{gather}1_{m_1}\times1_{m_2}=\psi_{m_1}\times\psi_{m_2}=1_{m_1+m_2}\nonumber\\1_{m_1}\times\psi_{m_2}=\psi_{m_1+m_2}\nonumber\\1_{m_1}\times\sigma_{m_2}=\psi_{m_1}\times\sigma_{m_2}=\sigma_{m_1+m_2}\nonumber\\\sigma_{m_1}\times\sigma_{m_2}=1_{m_1+m_2}+\psi_{m_1+m_2}.\end{gather} 

The electronic quasiparticle is the composition $\psi_{\mathrm{el}}=e^{-i4\phi_R}\gamma_L$ so that it is fermionic and has electric charge $-1$ in units of $e$. Since electron is the fundamental building block of the system, locality of $\psi_{\mathrm{el}}$ only allows primary fields $X$ that have trivial monodromy $\mathcal{M}^{X,\psi_{\mathrm{el}}}=1$ with the electron. As a result, this restricts $1_m,\psi_m$ to even $m$ and $\sigma_m$ to odd $m$. Lastly, the coupled wire models constructed later will involve the Pfaffian channels that propagate in both forward and backward directions. The backward case is denoted by $\overline{\mathrm{Pfaffian}}$, whose Lagrangian density is the time reversal of \eqref{Pfaffian}, i.e.~replacing $R\leftrightarrow L$, $i\leftrightarrow-i$ and $\partial_t\leftrightarrow-\partial_t$. 

\subsubsection{Gluing and splitting}\label{sec:gluing}

\begin{figure}[htbp]
	\centering\includegraphics[width=0.3\textwidth]{glueingsplitting}
	\caption{Gluing and splitting a pair of chiral Pfaffian 1D channels into and from a chiral Dirac channel.}\label{fig:glueingsplitting}
\end{figure}

A pair of co-propagating Pfaffian conformal field theory can be "glued" together into a single chiral Dirac electronic channel. Consider the decoupled pair $\mathcal{L}_0=\mathcal{L}_{\mathrm{Pfaffian}}^A+\mathcal{L}_{\mathrm{Pfaffian}}^B$, where $\mathcal{L}_{\mathrm{Pfaffian}}^{A/B}$ is the Lagrangian density of one of the two Pfaffian conformal field theory labeled by $A,B$. The pair of Majorana fermions can compose an electrically neutral Dirac fermion $d_L=(\gamma^A_L+i\gamma^B_L)/\sqrt{2}$, which can then be bosonized $d_L\sim e^{i\phi^\sigma_L}$, for $\phi^\sigma_L$ the chiral $\overline{U(1)_{1/2}}$ boson. Bosonization can be thought of as writing a fermion in terms of a boson, which usually ends up with the form $\psi\sim e^{i\phi}$, and you can rewrite the Lagrangian with this identity, although you cannot simply plug this in because you must be careful about normal ordering. For more detail see ~\cite{Senechal99} The bare Lagrangian now becomes the multi-component $U(1)_4^A\otimes U(1)_4^B\otimes\overline{U(1)_{1/2}}$ boson conformal field theory \begin{align}\mathcal{L}_0=\frac{1}{2\pi}\partial_t\boldsymbol{\phi}^TK\partial_x\boldsymbol{\phi}+\partial_x\boldsymbol{\phi}^TV\partial_x\boldsymbol{\phi},\label{881}\end{align} where $\boldsymbol{\phi}=(\phi_R^A,\phi_R^B,\phi^\sigma_L)$, $K$ is the $3\times3$ diagonal matrix $K=\mathrm{diag}(8,8,-1)$, and $V$ is some non-universal velocity matrix. A primary field is a vertex operator $e^{i{\bf m}\cdot\boldsymbol{\phi}}$ labeled by an integral vector ${\bf m}=(m^A,m^B,\tilde{m})$. It carries conformal spin $h_{\bf m}={\bf m}^TK^{-1}{\bf m}/2$ and electric charge $q_{\bf m}={\bf t}^TK^{-1}{\bf m}$ in units of $e$, where ${\bf t}=(2,2,0)$ is the charge vector. The Haldane criterion can be used to find backscattering terms that gap out excitations~\cite{Haldane95}. The algorithm for finding them first finds null vectors, i.e.~${\bf n}^TK{\bf n}=0$.  As ${\bf n}=(1,-1,4)$ is an electrically neutral null vector (${\bf t}\cdot{\bf n}=0$), it corresponds to the charge $U(1)$ preserving backscattering coupling \begin{align}\delta\mathcal{H}=-u\cos\left({\bf n}^TK\boldsymbol{\phi}\right)=-u\cos\left(8\phi^A_R-8\phi^B_R-4\phi^\sigma_L\right)\label{glueingH}\end{align} that gaps and annihilates a pair of counter-propagating boson modes. The interacting Hamiltonian can also be expressed in terms of many-body backscattering of the Pfaffians' primary fields \begin{align}\delta\mathcal{H}=-u:\left(d_L^\dagger d_R\right)^4:+h.c.\end{align} where $d_R=1_2^A1_{-2}^B$ is the electrically neutral Dirac fermion composed of the pair of oppositely charged semions in the two Pfaffian sectors.

In strong coupling, the gapping Hamiltonian introduces an interacting mass and the ground state expectation value $\langle\Phi\rangle=n\pi/2$, for $n$ an integer and $\Phi=2\phi^A_R-2\phi^B_R-\phi^\sigma_L$. In low energy, it leaves behind the chiral boson combination $\tilde\phi_R=2\phi_R^A+2\phi_R^B$, which has trivial operator product (i.e.~commutes at equal time) with the order parameter $\Phi$. The low-energy theory after projecting out the gapped sectors becomes \begin{align}\mathcal{L}_0-\delta\mathcal{H}\longrightarrow\mathcal{L}_{\mathrm{Dirac}}=\frac{1}{2\pi}\partial_t\tilde\phi_R\partial_x\tilde\phi_R+v(\partial_x\tilde\phi_R)^2\end{align} which is identical to the bosonized Lagrangian density of a chiral Dirac fermion. For instance, the vertex operator $\psi_R^{\mathrm{el}}\sim e^{i\tilde\phi_R}\sim 1_2^A1_2^B$ has the appropriate spin and electric charge of an electronic Dirac fermion operator ($h=1/2$ and $q=1$ in units of $e$). Notice that the vertex operator $e^{i\tilde\phi_R/2}$ has $-1$ monodromy with the local electronic $\psi_R^{\mathrm{el}}$ and therefore is not an allowed excitation in the fermionic theory.

Notice that the gluing potential \eqref{glueingH} facilitates an anyon condensation process~\cite{BaisSlingerlandCondensation}, where the maximal set of mutually local neutral bosonic anyon pairs \begin{align}\begin{array}{*{20}c}1_{4m}^A1_{-4m}^B,\psi_{4m}^A\psi_{-4m}^B,\\\psi_{4m+2}^A1_{-4m-2}^B,1_{4m+2}^A\psi_{-4m-2}^B,\sigma_{4m+1}^A\sigma_{-4m-1}^B\end{array}\label{condensebosons}\end{align} is condensed, where $m$ is an arbitrary integer. All primary fields that are non-local (i.e.~with non-trivial monodromy) with any of the condensed bosons in \eqref{condensebosons} are confined. Any two primary fields that differ from each other by a condensed boson in \eqref{condensebosons} are now equivalent. The condensation therefore leaves behind the electronic Dirac fermion \begin{align}\psi^{\mathrm{el}}_R=\psi^A_4\equiv\psi^B_4\equiv1_2^A1_2^B\end{align} and its combinations. 

\begin{figure}[htbp]
	\centering\includegraphics[width=0.5\textwidth]{fractionalization}
	\caption{Schematics of splitting a chiral Dirac channel into a pair of Pfaffian channels.}\label{fig:fractionalization}
\end{figure}

On the other hand, a chiral Dirac channel can be decomposed into a pair of chiral Pfaffian channels (see figure~\ref{fig:fractionalization} for a summary). The first problem is that the Pfaffian has many more degrees of freedom then a Dirac fermion. Perhaps from some channel re-construction, an additional pair of counter-propagating Dirac modes is appended to the chiral Dirac channel. This can be realized by pulling a parabolic electronic/hole band from the conduction/valence band to the Fermi level, or introducing non-linear dispersion to the original chiral channel. In low-energy, the three Dirac fermion modes can be bosonized $\psi^{1,2}_R\sim e^{i\tilde\phi_R^{1,2}}$, $\psi_L\sim e^{-i\tilde\phi_L}$ and they are described by the multicomponent boson Lagrangian \begin{align}\widetilde{\mathcal{L}}_{\mathrm{Dirac}}=\frac{1}{2\pi}\partial_t\widetilde{\boldsymbol\phi}^T\tilde{K}\partial_x\widetilde{\boldsymbol\phi}+\partial_x\widetilde{\boldsymbol\phi}^T\tilde{V}\partial_x\widetilde{\boldsymbol\phi}\label{3Dirac}\end{align} for $\widetilde{\boldsymbol\phi}=(\tilde\phi_R^1,\tilde\phi_R^2,\tilde\phi_L)$, $\tilde{K}$ is the diagonal matrix $\tilde{K}=\mathrm{diag}(1,1,-1)$, and $\tilde{V}$ is some non-universal velocity matrix. A general composite excitation can be expressed by a vertex operator $e^{i{\bf m}\cdot\widetilde{\boldsymbol\phi}}$, for ${\bf m}$ an integral 3-vector, with spin $h_{\bf m}=|{\bf m}|^2/2$ and electric charge $q_{\bf m}={\bf m}^T\tilde{K}\tilde{\bf t}$ in units of $e$, where $\tilde{\bf t}=(1,1,1)$ is the charge vector.

Next perform a {\em fractional} basis transformation \begin{align}\begin{array}{*{20}l}\phi^\rho_R=\tilde\phi^1_R+\tilde\phi^2_R+\tilde\phi_L\\\phi^\sigma_R=\tilde\phi^1_R-\frac{1}{2}\tilde\phi^2_R+\frac{1}{2}\tilde\phi_L\\\phi^\sigma_L=\tilde\phi^1_R+\frac{1}{2}\tilde\phi^2_R+\frac{3}{2}\tilde\phi_L\end{array}.\label{fracbasistrans0}\end{align} This follows the transformation rules from equations ~\ref{eq:transform1}~--~\ref{eq:transform5}. While the $\tilde{K}$ matrix is invariant under the transformation, the charge vector changes to $\tilde{\bf t}\to(1,0,0)$. $\psi^\rho_R\sim e^{i\phi^\rho_R}$ is the local electronic Dirac fermion that carries spin $1/2$ and electric charge $e$, and $d_{R/L}\sim e^{i\phi^\sigma_{R/L}}$ are counter-propagating electrically neutral Dirac fermions. As the $\tilde{K}$ matrix is still diagonal, these fermions have trivial mutual $2\pi$-monodromy and are local with respect to each other. However, it is important to notice that the neutral Dirac fermions $d_{R/L}$ actually consist of fractional electronic components.

Now consider the two $R$-moving Dirac channels. By pairing the Dirac fermions, they form two independent $SU(2)_1$ Kac-Moody current operators~\cite{bigyellowbook}. This is defined by the operator product expansion in equation~\ref{SU2algebra}. \begin{align}J_3^{A/B}(z)&=i2\sqrt{2}\partial_z\phi^{A/B}_R(z)\label{SU2current}\\J_\pm^{A/B}(z)&=\frac{J_1^{A/B}(z)\pm iJ_2^{A/B}(z)}{\sqrt{2}}=e^{\pm i4\phi^{A/B}_R(z)}\nonumber\end{align} where $4\phi^A_R=\phi^\rho_R+\phi^\sigma_R$ and $4\phi^B_R=\phi^\rho_R-\phi^\sigma_R$. Both $SU(2)_1$ sectors are electrically charged so that the bosonic vertex operators $J_\pm^{A/B}$ carries charge $\pm e$. They obey the $SU(2)$ current algebra at level 1 \begin{align}J^\lambda_{\mathsf{i}}(z)J^{\lambda'}_{\mathsf{j}}(w)=\frac{\delta^{\lambda\lambda'}\delta_{\mathsf{ij}}}{(z-w)^2}+\sum_{\mathsf{k}=1}^3\frac{i\sqrt{2}\delta^{\lambda\lambda'}\varepsilon_{\mathsf{ijk}}}{z-w}J^\lambda_{\mathsf{k}}(w)+\ldots\label{SU2algebra}\end{align} for $\lambda,\lambda'=A,B$. It is crucial to remember that $J_\pm^A\sim\psi^\rho_Rd_R$ and $J_\pm^B\sim\psi^\rho_Rd_R^\dagger$ contains the fractional Dirac components $d_R$. Thus, the primitive local bosons are actually pairs of the current operators, i.e.~$e^{i8\phi^{A/B}_R}$. Equivalently, this renormalizes the compactification radius of the boson $4\phi^{A/B}_R$ so that in a closed periodic space-time geometry, the electronic Cooper pair combinations such as the charge $2e$ local operators  \begin{gather}e^{i8\phi^A_R}=e^{i(4\tilde\phi^1_R+\tilde\phi^2_R+3\tilde\phi_L)}\sim(\psi^1_R)^4\psi^2_R(\psi_L^\dagger)^3\nonumber\\e^{i8\phi^B_R}=e^{i(3\tilde\phi^2_R+\tilde\phi_L)}\sim(\psi^2_R)^3\psi_L^\dagger\end{gather} are required to be periodic. The incorporation of anti-periodic boundary condition for $J_\pm^{A/B}=e^{\pm i4\phi^{A/B}_R}$ results in the $\mathbb{Z}_2$-orbifold theory~\cite{Ginsparg88,DijkgraafVafaVerlindeVerlinde99} $U(1)_4=SU(2)_1/\mathbb{Z}_2$ for both $A$ and $B$ sectors. Orbifolding usually results in new "twist" fields", or fields that when braided with the object that is can have antiperiodic boundary conditions, yeilds the -1 required. For instance, the primitive twist fields are given by $e^{\pm i\phi^{A/B}_R}$, which have $-1$ monodromy phase with $J_\pm^{A/B}$. 

At this point, including the $L$-moving neutral Dirac sector, the muticomponent boson $\boldsymbol\phi=(\phi^A_R,\phi^B_R,\phi^\sigma_L)$ described by the Lagrangian \eqref{881} has been recovered . Lastly, simply decompose the remaining neutral Dirac into Majorana components, $d_L=(\gamma^A_L+i\gamma^B_L)/\sqrt{2}$. The $A$ and $B$ Pfaffian sectors can then be independently generated by the charged $U(1)_4$ boson $\phi^{A/B}_R$ and the neutral Majorana fermion $\gamma^{A/B}_L$. As a consistency check, the charge $e$ fermionic (normal ordered) combinations defined in \eqref{Pfaffianfields} \begin{align}\psi_4^A&\sim e^{i4\phi^A_R}\gamma_L^A\sim e^{i\tilde\phi^1_R}+e^{i(3\tilde\phi^1_R+\tilde\phi^2_R+3\tilde\phi_L)}\label{psi4def1}\\\psi_4^B&\sim e^{i4\phi^B_R}\gamma_L^B\sim e^{i(-\tilde\phi^1_R+\tilde\phi^2_R-\tilde\phi_L)}-e^{i(\tilde\phi^1_R+2\tilde\phi^2_R+2\tilde\phi_L)}\nonumber\end{align} are in fact local quasi-electronic. %(The minus sign in the bosonized expression for $\psi_4^A$ comes from the Klein factors defined later in \eqref{ETcomm0} and \eqref{ETcomm1}).

Unlike in the gluing case where there is a gapping Hamiltonian \eqref{glueingH} that pastes a pair of Pfaffians into a Dirac, here in the splitting case there has been some kind of fractional basis transformation that allows an expression of a Dirac channel as a pair of Pfaffians. In fact, one can check that the energy-momentum tensor of the Dirac theory \eqref{3Dirac} is identical to that of a pair of Pfaffians \eqref{Pfaffian}. However, this does not mean the Pfaffian primary fields are natural stable excitations. In fact, as long as there is a pair of co-propagating Pfaffian channels, all primary fields except the non-fractionalized electronic ones are unstable against the gluing Hamiltonian $\delta\mathcal{H}$ in \eqref{glueingH} and are generically gapped. In order for the Pfaffian conformal field theory to be stablized, one has to suppress $\delta\mathcal{H}$. A possible way is to somehow spatially separate the pair. This issue is addressed in the coupled wire paper below using many-body interaction in the coupled wire model of a Dirac semimetal (or the particle hole symmetric Pfaffian fractional quantum Hall state in Ref.~\onlinecite{KaneSternHalperin17}).




\subsection{What is an Topological Insulator?}
 One description of insulators comes from the band theory of solids. First a lattice needs to be defined. Say there is an atom at the origin. Then another atom at some fixed point $\vec{v}$. If an atom is placed at every point in the set $L= \{\vec{x} = a\vec{v} \quad | \quad a \in \mathbb{Z} \}$ it is a 1-dimensional bravais lattice. If there are n points $\vec{v_i}$ and atoms at every point in the set $L= \{\vec{x} = \sum_i a_i\vec{v_i} \quad | \quad a_i \in \mathbb{Z} \}$ it is an n dimensional bravais lattice. These vectors are called lattice vectors. These lattices have what is called a unit cell, which is a volume of space that when translated by the vectors $\vec{v_i}$, will recreate the entire space. There can be m atoms in the unit cell as well and they will be translated instead of just a single atom. These are called bravais lattices with an m-point basis. These structures describe all crystalline solids! Notably that is not all solids. Each and every bravais lattice comes with certain symmetries, such as translation, but possibly mirror, or rotation, or a combination. These symmetries affect what wavefunctions can live on the lattice.

Once there is a lattice, there is Bloch's Theorem. First, assume there is a wavefunction that is the eigenstate of all translation operators $T_{n_1,n_2,...n_d}$ where d is the dimension of our lattice, and the operator translates wavefunctions by $\sum_i n_i \vec{a_i}$
\begin{align}
\psi(\vec{r}+\vec{a_j})=C_j\psi(\vec{r})
\end{align}
In fact it is more useful to let $C_j=e^{2 \pi i \theta_j}$. Define $\vec{k} = \sum_i \theta_i\vec{b_i}$ where $\vec{b_i}$ are so called "reciprocal lattice vectors" meaning, $\vec{a_i}\cdot \vec{b_j}=2 \pi \delta_{ij}$. Finally define the bloch wave $u(\vec{r})=e^{-i \vec{k}\cdot \vec{r}}\psi(\vec{r})$. Then,
\begin{align}
u(\vec{r}+\vec{a_i})=e^{-i \vec{k}\cdot (\vec{r}+\vec{a_i})}\psi(\vec{r}+\vec{a_i})=e^{-i \vec{k}\cdot \vec{r}}e^{-2 \pi i \theta_i }e^{2 \pi i \theta_i}\psi(\vec{r})=u(\vec{r})
\end{align}
This means $u(\vec{r})$ has the same periodicity as the crystal. Now, \underline{\textbf{if}} the Hamiltonian H has these translation symmetries, it must commute with them. If is emphasized because the translation operator acts on one particle, and sometimes H does not. If they commute, H and the translation operators share an eigenbasis. In this basis, with fixed translation eigenvalues i.e. fixed k, the eigenfunctions are of the form $\psi_k(\vec{r})=e^{i \vec{k} \cdot \vec{r}} u(\vec{r})$. 

Now to find the energy of such a particle, integrate  
\begin{align}
\int_r u(\vec{r}) e^{-i \vec{k} \cdot \vec{r}} H e^{i \vec{k} \cdot \vec{r}} u(\vec{r}) = E(\vec{k})
\end{align}

Now this means that the wavefunctions can be simplified to single unit cell. Moreover, if the k vector is changed by a reciprocal lattice vector, the $e^{i \vec{k} \cdot \vec{r}}$ term does not change, meaning the wave functions and energies do not change. This means the definition is somewhat redundant and if one restrict themselves to a "unit cell" in k space, called the Brillouin zone, all of the energies can be derived. There are usually more then one of these wavefunctions, since there is no reason for there to be only one electron in a unit cell. These $E(\vec{k})$ then make up several "bands".

Most solids can be described by this band theory. Since these bands describe all of the energy levels, and they are usually filled by electrons, at zero temperature they will be filled to the energy of the highest energy electron also known as the Fermi energy. If that Fermi energy is crossing a band, that means that an electron can travel up the band, i.e. change momentum for an infinitesimal energy cost. Since that energy is usually available via thermal fluctuations, this makes a conductor. If there is no band at the Fermi energy, then there is a finite energy cost to jump from the highest filled state to the lowest empty state. If that cost is large it is an insulator, if it is small, it is a semiconductor. This is called the energy gap. 

An insulator then has been defined using an energy gap. The functions $E(k)$ for an insulator can be looked at as maps from the Brillouin zone, which in n dimensions is an n-torus, to the space of $(\mathbb{R}-\{0\}) \oplus T_n $. Two insulators made of different atoms in a sense belong to the same phase. The Hamiltonian can be mathematically changed to go from one insulator to another, without closing the energy gap. This defines topological equivalence classes. For a broader equivalence class, it can be defined that insulators with a different number f bands are also equivalent. By definition, the transition between two non equivalent topological insulators would be conducting. Naively, one might think all insulators would be equivalent. 

One of the most classic examples of a topological insulator is the shown with the quantum Hall effect. In a 2-D gas of electrons in a strong perpendicular magnetic field, electrons move in small circles. Independently, each electron is a 2-D quantum harmonic oscillator, an will have an energy of $e_n = \hbar \omega_c(n + 1/2)$ with $\omega_c = eB/m$ being the cyclotron frequency. If an integer number of these bands are completely filled, there will be an energy gap.

Unlike a typical insulator however, if an electric field in the plane is applied, the orbits will start to drift, and the magnetic field will create a transverse motion as well. This would be expected with the classical Hall effect as well. The interesting part is if the magnetic field is large enough to make sure all the electrons fill N energy levels, the Hall conductance becomes quantized as $\sigma_{xy}=Ne^2/h$. If the Hamiltonian is changed adiabatically to get to a different number of filled energy levels, at some point there will be a partially filled/empty energy level making the gap = 0. This has been used to measure $e^2/h$ to one part
in $10^9$ (von Klitzing, 2005). Even more interesting, if there is a boundary along the gas, this is a transition between a topological insulator (the gas) and a trivial insulator (the vacuum). There is indeed a conducting edge mode there! This can be understood by the orbits of the electrons bouncing off the edge. Even more interesting, the edge mode is chiral, meaning they only travel in one direction. This is known as the chiral anomaly in high energy, or the Nielsen-Ninomiya theorem in condensed matter. As is to be expected in topology, there is a type of bulk boundary correspondence.

Now what is the difference between this an normal insulator? Well for starters a good way to know any phase from another is with a order parameter. This in general is just some function that changes discontinuously during a phase transition. For example, the average distance from lattice sites changes discontinuously as a solid melts. Typically these are local, meaning they can be measured in some small vicinity. Topological insulators though are differentiated using a global order parameter, meaning a correlation function that cannot be measured locally. For this example, the global order parameter is a topological invariant known as the Chern number, which describes maps from the torus to H(k). This can be understood using fiber bundles in mathematics, but can also be thought of using the berry phase. If there is a state $|u_m(k)>$, and k is changed along some loop, there will be a phase which is the line integral of $A_m = i <u_m(k)|\nabla_k|u_m(k)>$. This by stokes theorem can be rewritten in terms of the berry flux $F_m=\nabla \times A_m$. The Chern invariant is found by integrating this over the Brillouin zone
\begin{align}
 n=\frac{1}{2\pi}\int_{BZ}dk^2 F_m, \quad \sigma_{xy}=\sum n
\end{align}
where the sum is over all occupied bands where $\sigma_{xy}$ is called the total chern number. This was shown to be the same as the quantum Hall condunctance by TKNN~\cite{TKNN}. This integral needs to know u(k) completely, so it is not local. This is nothing more then the Gauss-Bonet theorem in disguise, which gives an integral for the genus of a surface, and describes a map from the torus onto a hilbert space.

\subsection{What are Dirac/Weyl semimetals?}

A good way to study topological insulators is to start at a known conducting state, and see if the Hamiltonian can be tuned into distinct insulating states. This is motivation here for the study of Dirac/Weyl semimetals. A Dirac/Weyl semimetal is a band theoretic model which close to the Fermi energy follows the massless Dirac/Weyl equation. Graphene, a honeycomb lattice of carbon atoms, is a prime example of this. The honeycomb lattice has a 2 point basis, so since the unit cell has at least 2 atoms in it, there will be at least two bands. The $p_z$ orbital bands are the closest to the Fermi energy, so the only considering the $p_z$ orbitals is a first approximation. If a tight binding model is used, meaning the electrons wavefunctions are approximately localized, with some probability of tunneling to the next site, the Hamiltonian is just hopping terms from A atoms to B atoms.
\begin{align}
H=\sum_{<r,r'>,l,l'}t^{l,l'}_{r,r'}c^{l\dag}_rc^{l'}_{r'} +h.c.
\end{align}
where <l,l'> are nearest neighbor atoms A or B, and r and r' are lattice unit cell positions, and the c's are electron creation and annihilation operators. The quantum states here are superpositions of $c^{l\dag}_r$ operators on the vacuum. It can be represented by a complex column vector where the length is the number of distinct $c^\dag$'s. After taking the Fourier transform of c, this will become 
\begin{align}
H=\int_{BZ}\frac{dk^2}{(2\pi)^2} 
\begin{pmatrix}
c^{A\dag}_k & c^{B\dag}_k
\end{pmatrix}
H(k)
\begin{pmatrix}
c^{A}_k \\
c^{B}_k
\end{pmatrix} 
\end{align}
where H(k) is a 2x2 matrix dependent on the t's. The eigenfunctions commute with translation because of Bloch's theorem. H(k) can be solved independently for each k on the two vector quantum states where $\begin{pmatrix}
	1 \\
	0
\end{pmatrix} = c^{A\dag}_k|0>$ and  $\begin{pmatrix}
0 \\
1
\end{pmatrix} = c^{B\dag}_k|0>$
H(k) is a 2x2 hermitian matrix, so it can be broken down in terms of pauli matricies
\begin{align}
H(k)=\vec{h(k)}\cdot \vec{\sigma}
\end{align}
where $\vec{h(k)} = (h_x(k),h_y(k),h_z(k))$ are real valued functions and $\vec{\sigma} = (\sigma_x,\sigma_y,\sigma_z)$. It is assumed there is no part proportional to the identity, since this would just shift the spectrum up or down. Any symmetry S that the Hamiltonian has means that $SHS^{-1}=H$, where S is defined by how it changes the quantum state. To analyze these symmetries the operation of them on H and k have to be considered separately. This can restrict the form of H(k).

Here is the graphene model in more detail. 

\begin{align}
H=\sum_{r} t c^{A\dag}_{r+\delta_1}c^B_r+ t c^{A\dag}_{r+\delta_2}c^B_r+ t c^{A\dag}_{r+\delta_3}c^B_r +h.c.
\end{align}
\begin{figure}
	\centering
	\includegraphics[width=0.5\linewidth]{images/HoneycombGreen}
	\caption{cite http://inspirehep.net/record/1256684/files/HoneycombGreen.png}
	\label{fig:honeycombgreen}
\end{figure}


where $a_{1,2} =a(\sqrt{3}/2),\pm1/2)$, and $\delta_3 =a(0,-1) $, where a is the lattice spacing as in fig~\ref{fig:honeycombgreen}. Notice there is no A to A and B to B terms, since these come from next nearest neighbor hopping at first order, and inversion symmetry enforces that they be equal. If they are the equal, it simply adds an identity piece to H(k).

Taking the Fourier transform, $c_r = \int_k \frac{dk^2}{(2\pi)^2}e^{-ikr} c_k$, each of the $+\delta$ terms simply contribute an $e^{\pm i k \cdot \delta}$ depending on whether there was a dagger or not. Now
\begin{align}
H(k)= 
\begin{pmatrix}
0 & h(k) \\
h^*(k) & 0
\end{pmatrix} \quad h(q) = \sum_i e^{-ik\delta_i} = e^{-ik_ya} + e^{ik_xa\sqrt{3}/2+ik_ya/2}+e^{-ik_xa\sqrt{3}/2+ik_ya/2} \\
h(q) = cos(k_ya)-i sin(k_ya) +(cos(k_ya/2)+i sin(k_ya/2))2cos(k_xa\sqrt{3}/2)
\end{align}
This means that $h_x(k) = cos(k_ya)+2cos(k_xa\sqrt{3}/2)cos(k_ya/2)$ and $h_y(k) = sin(k_ya)-2cos(k_xa\sqrt{3}/2)sin(k_ya/2)$. Notice $h_x(k) = 0$ when $k = \pm K = \frac{2\pi}{3a}(\pm1/\sqrt{3},-1)$, since $h_x(\pm K) = cos(\frac{2\pi}{3})+2cos(\pm\frac{\pi}{3})cos(\frac{\pi}{3})=0$, and $h_y(\pm K) = sin(\frac{2\pi}{3})-2cos(\pm\frac{\pi}{3})sin(\frac{\pi}{3})=0$.

This degeracy in general is protected by symmetries. First there is a time teversal symmetry T. In general, given any time dependent wave function, the time dependent pieces are a series of phases $e^{iEt/\hbar}$ onto time independent pieces. Changing the sign of t is equivalent here to complex conjugation, so T can be represented by the complex conjugation operator K. This representation is dependent on the basis. 
An important point here is that T is anti-unitary. In general under a symmetry S, any correlation function $|<Sa|Sb>| = |<a|b>|$. This means that $|<a|S^TS|b>| =|<a|b>|$ or $|<b|a>|$. Time reversal does the second. There is a simple theorem that says antiunitary symmetries are unitary symmetries with the complex conjugate operator. In this basis, creation and annihilation operators in real space map to real valued matricies, so K leaves them alone. It flips the sign of k, so it sends $c_k$ to $c_{-k}$. We can use the same basis of $\begin{pmatrix}
1 \\
0
\end{pmatrix} = c^{A\dag}_k|0>$ and  $\begin{pmatrix}
0 \\
1
\end{pmatrix} = c^{B\dag}_k|0>$
\begin{align}
THT^{-1}&=\int_{BZ}\frac{dk^2}{(2\pi)^2} 
\begin{pmatrix}
c^{A\dag}_k & c^{B\dag}_k
\end{pmatrix}
TH(k)T^{-1}
\begin{pmatrix}
c^{A}_k \\
c^{B}_k
\end{pmatrix}
 \\
&= \int_{BZ}\frac{dk^2}{(2\pi)^2} 
\begin{pmatrix}
c^{A\dag}_{k} & c^{B\dag}_{k}
\end{pmatrix}
KH(-k)K
\begin{pmatrix}
c^{A}_{k} \\
c^{B}_{k}
\end{pmatrix} = H
\end{align}

This implies $H^*(-k) = H(k)$. A symmetry can be thought of as both acting on H, and on k.

There is also an inversion symmetry P around the center of a unit cell which sends r to -r. Inversion P sends A sites to B sites, and B to A, while also flipping the sign of k. $\sigma_x$ flips the atoms, so P can be represented by $\sigma_x$. 

The Inversion operator flips the atoms and r coordinate of the real space c operators. This ends up sending $c^A_k = \int_r e^{-ikr}c^A_r$ to $c^B_{-k}$, and similarly with the rest. This has the effect of making the Hamiltonian
\begin{align}
PHP^{-1}&=\int_{BZ}\frac{dk^2}{(2\pi)^2} 
\begin{pmatrix}
c^{A\dag}_k & c^{B\dag}_k
\end{pmatrix}
PH(k)P^{-1}
\begin{pmatrix}
c^{A}_k \\
c^{B}_k
\end{pmatrix} \\
&= \int_{BZ}\frac{dk^2}{(2\pi)^2}
\begin{pmatrix}
c^{A\dag}_k & c^{B\dag}_k
\end{pmatrix}
\sigma_x H(-k) \sigma_x
\begin{pmatrix}
c^{A}_k \\
c^{B}_k
\end{pmatrix}
= H
\end{align}
Inversion makes sure $\sigma_x H(-k) \sigma_x =H(k)$. 

Together these constrain the Hamiltonian. By time reversal, $Kh_x(k) \sigma_xK = h_x(-k)\sigma_x$, which implies $h_x(-k)=h_x(k)$. Inversion makes $\sigma_x (h_x(k) \sigma_x) \sigma_x =h_x(-k)\sigma_x$ which implies the same thing. This means $h_x(k)$ is even in k. For $h_y(k)$, $K h_y(k) \sigma_y K =-h_y(-k)\sigma_y= h_y(k)\sigma_y$. This implies $h_y(k) = -h_y(-k)$. Inversion this time implies  $\sigma_x h_y(k) \sigma_y \sigma_x =-h_y(-k)\sigma_y=  h_y(k)\sigma_y$, which again is the same thing, making $h_y$ odd under k.

Now the important part. By time reversal, $Kh_z(k) \sigma_zK = h_z(-k)\sigma_z$, which implies $h_z(-k)=h_z(k)$. Inversion makes $\sigma_x (h_z(k) \sigma_z) \sigma_x = -h_z(-k)\sigma_z$, which implies $-h_z(-k)=h_z(k)$. Now we have $-h_z(-k)=h_z(-k)$, which implies $h_z(k)$ is zero!
This means that there is no $\sigma_z$ component. Then by squaring the H operator,
\begin{align}
H^2(k)= \sigma_x^2 h_x^2(k) + \sigma_y^2 h_y^2(k) \\
H^2(k)= h_x^2(k) + h_y^2(k) \\
E(k)=\pm\sqrt{h_x^2(k) + h_y^2(k)}
\end{align}.
Since there is no $\sigma_z$ piece, the only changes that could be added is parts to $h_x$ or $h_y$. Since around K or K' point there is $E(K+q)=\sqrt{q_x^2+q_y^2}$, all that can be done is add a constant, which would just shift the Dirac points, or add higher order pieces, which still go to zero. Now if one of the symmetries is broken, there can be d $\sigma_z$ pieces, which can gap out the system. There are topologically distinct ways to do this.

Consider a small constant $M\sigma_z$ term added to H(k), which can be done in real space by adding $Mc^{A\dagger}_rc^A_r - Mc^{B\dagger}_rc^B_r +h.c$ terms to the sum. This breaks inversion symmetry, and is physically making the atoms on the A site and B site different. In the limit of large M, this is basically binding all the atoms to one site. That is a model of a trivial insulator.

Haldane showed that there is a topologically nontrivial phase here, if time reversal is borken but not inversion. The intuition here being if there is a term that can be added is $h_z(k)\sigma_z$ where $h_z(k)=-h_z(-k)$ to preserve time reversal. In the limit that k is close to K
\begin{align}
H(q=(k-K))= \sigma_x h_x(q) + \sigma_y h_y(q) + \sigma_z h_z(q)\\
H(q)= c(\sigma_x q_x + \sigma_y  q_y) + \sigma_z h_z(q)
\end{align}
For some constant c. If $h_z\sigma_z$ is time reversal symmetric, the Hamiltonian near -K is known as well. 

 Haldane added next nearest neighbor hopping to another 6 sites, such that all lattice symmetries stayed. This adds a piece onto H(k) as follows:
\begin{align}
H_2(k) =  2t_2 cos(\phi)\sum_i cos(k \cdot b_i)  +2t_2 sin(\phi) \sum_i sin(k \cdot b_i)
\end{align} 
where $b_1 = \delta_2 - \delta_3$, $b_2 = \delta_3 - \delta_1$, and $b_3 = \delta_1-\delta_2.$ This makes $h_z(k=\pm K) = \mp 3\sqrt{3}t_2sin(\phi)$. In the limit of being close to K or -K, it is the fact that this gap flips sign at $\pm K$ that makes the topology nontrivial.

The berry curvature integral ends up taking the form
\begin{align}
n = \frac{1}{4\pi}\int_{BZ}dk^2 \partial_{k_x}\hat{h}(k) \times \partial_{k_y}\hat{h}(k) \cdot \hat{h}(k) \\
\hat{h}(k) = \vec{h}(k)/|\vec{h}(k)|
\end{align} 
This counts the number of times $\hat{h}(k)$ wraps around the unit sphere. Notice this is only defined when there is a gap. In the trivial phase, the z component of $\hat{h}(k)$ is always positive. That means $\hat{h}(k)$ can't possibly wrap around the sphere, so n would be 0. In this Haldane phase however, $\hat{h}(\pm K) = (0,0,\mp 1)$, and in fact wraps around the sphere exactly once. This phase actually exhibits a quantum Hall conductance of $\sigma_{xy}=e^2/h$. If this model is on a semi-infinite plane with an edge, there will be a single edge mode which carries this charge. Since this map can't go from n=1 to 0 continuously, the only way to change phases would be to close the gap, i.e. make $\hat{h}(k)$ ill-defined. This is the definition of a topological insulator! 

\subsection{Topological insulators with spin}

Now, this was just a toy model, in real life electrons have spin, and this changes things rather drastically. This thesis will go through a model by Kane and Mele~\cite{KaneMele2D1,KaneMele2D2}, which is written pedagogically in Fradkin's book~\cite{Fradkinbook}. Now there are two copies of the exact same model, one for spin up and one for spin down. A gap can be opened using spin-orbit interactions.

In the Haldane model, the Chern invariant ended up being how many times some vector $\hat{h}(k)$ wrapped around the sphere. Based on the sign difference of that at the old degenerate points the number of times it wrapped around the sphere could be deduced. The degenerate point $(k_1,k_2)=(\pi,\pi)$, where $k_i$ are reciprocal lattice vectors, was sent to K' under time reversal and inversion. Those relation put constraints on the Hamiltonian. When spin is involved, there can be a new topological invariant, which can be nonzero without time reversal breaking. 

With spinful electrons, the previous spinless model needs to be doubled, one for spin up and one for spin down. One can simply be the time reversal copy of the other. Now if the first has a Chern number of c, the second will have a Chern number of -c. The total charge conductance cancels out, but there is still a "spin current", since spin ups move in one direction and the spin downs move in the other. The topological invariant associated with this can again be found out by looking at old degenerate points and symmetry constraints.

Now the T operator is described. An interesting fact here is $T^2 =-1 $ for fermion systems. This can understood from CPT symmetry. In euclidean space CPT is just a 180 degree rotation. That squared then is a 360 rotation, which is +1 for bosons and -1 for fermions. If $(CP)^2=1$ and it is assumed these operators commute, then $T^2=-1$. T has to be real valued by the operation seen above, and has a piece that's $\tau_x$ or $\tau_y$, to switch spins. The $\tau_x$ would give the wrong $T^2$, so $T = i\tau_yK$, and $T^2 = i\tau_yi\tau_yK^2 = i^2 = -1$.

In 2+1 d spinful systems, there is whats called a Kramer's degeneracy protected by just time reversal. If $T|\phi>= e^{i\theta}|\phi>$, meaning $|\phi>$ is an eigenstate of T, then $TT|\phi> = T e^{i\theta}|\phi> = e^{-i\theta}T|\phi>=e^{-i\theta}e^{i\theta}|\phi>= |\phi>$ which implies $T^2=+1$. The contrapositive means if $T^2=-1$ on a state $|\phi>$then $|\phi>$ is not an eigenstate of T. Then if time reversal symmetry acts on a state $T|\phi_1(k)> = |\phi_2(-k)>$ and since H and T commute,
\begin{align}
TH|\phi_1(k)> = TE_1(k)|\phi_1(k)> = E_1(k)|\phi_2(-k)> \\
HT|\phi_1(k)> = H|\phi_2(-k)> = E_2(-k)|\phi_2(-k)> \\
E_1(k)=E_2(-k)
\end{align} 
So if there is a time reversal invariant point, i.e. $k=-k+G$ where G is a reciprocal lattice vector, there is a degeneracy. Again what matters is a different sign on the "mass" at degenerate points. Then the topological invariant is discovered by looking at these degenerate points, or the matrix given by $w_{ij}(k) = <\phi_i(-k)|T|\phi_j(k)>$, This is of interest when k is time reversal invariant. If it is assume for a moment that there are only 2 bands, this w is nonzero if $i\ne j$. It can be guessed that the sign of the gapping term would be given by the sign of $ \sqrt{\det{w(K)}}$. This is not just a simple $\pm1$, so interestingly enough the Pfaffian is used, which remember squares to the determinant. In the 2 x 2 case, it is just $M_{12}$. Define $\delta(k) = \sqrt{\det{w(K)}}/pfaff(w(K)) = \pm 1$, or in the 2 x 2 case = $\pm sgn(w_{12})$ This function is of great importance later in this thesis. One issue is the square root, which leaves the sign ambiguous. This is a continuous function however, and the sign differences at different invariant points is defined. In 2+1 d there are four time reversal invariant momenta $Q_i$, so let $(-1)^\nu = \prod_i \delta(Q_i)$. This ends up being gauge invariant and $\nu$ is a topological invariant that carries two values $0,1\mod 2$. It is called time reversal polarization.

This is called a $Z_2$ index. The reason why this is no longer Z classified is that there could be non spin conserving terms that are still time reversal invariant. If there is N chiral modes on the spin up sector, there will be N chiral modes on the spin down sector. If there are backscattering terms, there will be a scattering matrix. If  there is a state
\begin{align}
|\psi_{L}>=\sum_{i=1}^N \alpha_{i,L}|\psi_\uparrow^L>+\beta_{i,L} T|\psi_\uparrow^L> \\
|\psi_{R}>=\sum_{i=1}^N \alpha_{i,R}|\psi_\uparrow^R>+\beta_{i,R} T|\psi_\uparrow^R>  
\end{align}
with $\alpha$'s for the incoming modes and $\beta$'s for the outgoing. The scattering matrix goes as 

\begin{align}
\begin{pmatrix}
\vec{\beta}_L \\
\vec{\beta}_R
\end{pmatrix}
=
\begin{pmatrix}
r & t\\
t' & r'
\end{pmatrix}
\begin{pmatrix}
\vec{\alpha}_L \\
\vec{\alpha}_R
\end{pmatrix}
\end{align}
Now S has to be unitary since it is a change of basis from incoming to outgoing states. Time reversal will send outgoing to incoming and ect. S on the states now gives $ST^2\vec{\beta}^*=\vec{\alpha}^*$. The incoming $\beta$ states previously had no T's, and now there are two. If $vec{\beta}$ is solved for then $\vec{\beta}=T^2S^T\vec{\alpha}$. If $T^2=1$, this means $S=S^T$ together with $S^{-1}=S^\dag$. In this case if t=t'=0, that just requires r and r' to satisfy the same set of conditions. Yet if $T^2=-1$ that means S is antisymmetric. Then if t=t'=0 that means r and r' are antisymmetric and unitary. If N is odd, that means it there is an odd dimension antisymmetric matrix, which always has a zero eigenvalue. This is impossible for a unitary matrix. This means not only that t and t' are not zero, but that one mode must have perfect transmission. So this topological index simply measures the parity of the number of helical modes.

This generalizes to 3+1 d. There are 8 invariant points in 3+1, and $(-1)^{\nu_0} = \prod_i \delta_i$, and 3 more invariants $(-1)^{\nu_k} = \prod_i \delta_i$ where the sum is now just 4 points on the xy, yz, or zx planes of the Brillouin zone. This is just treated the 3d topological insulator as a 2d one in different projections. These three are called weak topological insulators and the sum over 8 is called a strong topological insulator, since it turns out to be robust to disorder.

Here is a brief summary of what was seen in this section. Topological insulators are band insulators that cannot be deformed into the atomic limit insulator without closing the gap. Two models of this were presented that exhibit the quantum Hall effect and the quantum spin Hall effect. The symmetries constrain the Hamiltonian to have a certain form, and if starting from a semi-metal, a gap can be opened in distinct ways. A topological index being zero does not mean the phase is trivial. For instance, the quantum spin Hall effect has trivial Chern number. Even if all the indecies known were zero there might be nonzero ones that haven not been thought of yet. A conventional TI has a set of symmetries, so far at minimum time reversal and charge conjugation.
 
This sums up an introduction to tools used in the two papers described in the rest of this thesis. Classical, integer quantum Hall, and then fractional quantum Hall were described. Following that the specifics of a $\nu=1/2$ state were seen. Then using conformal field theory these fractional excitations can be analyzed. These fractional states can be written in terms of electronic operators and vice versa. Now using this machinery a few questions can be addressed. First, since a Dirac mode can be split into Pfaffian modes, if a Dirac semimetal can be described using these Dirac modes, they can be split into Pfaffian modes. Perhaps backscattering Pfaffian modes can provide a new type of gapping. This will be explored first. Next, the Pfaffian state can live on the surface state of a topological insulator. This could be generalized to fractional topological insulators. That is the second publication. What follows is a mostly a direct exert from these publications~\cite{RazaSirotaTeo17,SahooSirotaChoTeo17}.
 


