A rather extensive background is necessary to understand what this thesis addresses. Here a brief overview is given, along with many materials that are useful for further understanding. It is important to provide a form of motivation for this work before delving in to the mathematical details. This thesis is based on two papers, one describing surfaces of fractional topological insulators and the other on a coupled wire model for Dirac and Weyl semimetals. Starting with the former, 3+1 d fractional topological insulators support local excitations in the bulk that in general carry fractional charge and exotic exchange phases. The parton model breaks the electron creation operator into parts, which are deconfined in the bulk of the material. Gauge fluxes are necessary to make sure these parts are confined in vacuum, much like quarks that make up a proton are confined. This model provides a way to describe excitations in fractional topological insulators. The interactions of these excitations with different types of surfaces is examined. The resulting picture gives types of surface excitations in 3+1 d fractional topological insulators, to complement the understanding of conventional topological insulator surface states. Specifically models generalizing the conventional topological surface Pfaffian state with a new filling fraction of $1/2(2n+1)$ are described.
The second paper describes a model of Dirac and Weyl semimetals. This is of interest because these semimetals typically are transitions between different types of topological phases. It is known how to change the semimetal Hamiltonian to create topological insulators, and topological superconductors by breaking some of the symmetries in the semimetal. Here a way to add many body interactions that yields a new topological phase that preserves the semimetals symmetries is described. What is in common and the central goal of these two works is to describe new topological phases 