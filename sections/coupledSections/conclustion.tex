Dirac and Weyl (semi)metals have generated immense theoretical and experimental interest. On the experimental front, this is fueled by an abundant variety of material classes and their detectable \ARPES and transport signatures. On the theoretical front, Dirac/Weyl (semi)metal is the parent state that, under appropriate perturbations, can give birth to a wide range of topological phases, such as topological (crystalline) insulators and superconductors. In this work, we explored the consequences of a specific type of strong many-body interaction based on a coupled-wire description. In particular, we showed that (i) a 3D Dirac fermion can acquire a finite excitation energy gap in the many-body setting while preserving the symmetries that forbid a single-body Dirac mass, and (ii) interaction can enable an anomalous antiferromagnetic time-reversal symmetric topological (semi)metal whose low-energy gapless degrees of freedom are entirely described by a pair of non-interacting electronic Weyl nodes separated in momentum space. A brief conceptual summary was presented in section~\ref{sec:introsummary} and will not be repeated here. Instead, we conclude by discussing possible future directions.

%superconducting version and nodal systems
First, coupled wire constructions can also be applied in superconducting settings and more general nodal electronic systems. For example, a Dirac/Weyl metal can be turned into a topological superconductor~\cite{SchnyderRyuFurusakiLudwig08,Kitaevtable08,QiHughesRaghuZhang09} under appropriate intra-species (i.e.~intra-valley) $s$-wave pairing~\cite{QiWittenZhang13}. Pairing vortices host gapless chiral Majorana channels~\cite{QiWittenZhang13,GuQi15,LopesTeoRyu17}. An array of these chiral vortices can form the basis in modeling superconducting many-body topological phases in three dimensions. On the other hand, instead of considering superconductivity in the continuous bulk, inter-wire pairing can also be introduced in the coupled Dirac wire model and lead to new topological states~\cite{ParkTeoGilbertappearsoon}.

Dirac/Weyl (semi)metals are a specific type of nodal electronic matter. For example, nodal superconductors were studied in states with dx$^2$-y$^2$ pairing~\cite{RyuHatsugaiPRL02}, He$^3$ in its superfluid A-phase~\cite{Volovik3HeA,Volovikbook}, and non-centrosymmetric states~\cite{SchnyderRyuFlat,BrydonSchnyderTimmFlat}. Weyl and Dirac fermions were generalized in time reversal and mirror symmetric systems to carry $\mathbb{Z}_2$ topological charge~\cite{morimotoFurusakiPRB14}. General classification and characterization of gapless nodal semimetals and superconductors were proposed~\cite{Sato_Crystalline_PRB14,ZhaoWangPRL13,ZhaoWangPRB14,ChiuSchnyder14,matsuuraNJP13,Volovikbook,RMP,HoravaPRL05}. It would be interesting to investigate the effect of strong many-body interactions in general nodal systems.

Second, in section~\ref{sec:DiracSemimetal}, we described a coarse-graining procedure of the coupled wire model that resembles a real-space renormalization and allows one to integrate out high energy degrees of freedom. While this procedure was not required in the discussions that follow because the many-body interacting model we considered was exactly solvable, it may be useful in the analysis of generic interactions and disorder. The coarse-graining procedure relied on the formation of vortices, which were introduced extrinsically. Like superconducting vortices, it would be interesting as a theory and essential in application to study the mechanism where the vortices of Dirac mass can be generated dynamically. To this end, it may be helpful to explore the interplay between possible (anti)ferromagnetic orders and the spin-momentum locked Dirac fermion through antisymmetric exchange interactions like the Dzyaloshinskii-Moriya interaction~\cite{Dzyaloshinsky58,Moriya60}.

%topological order, threefold lattice and alternative fractionalization
Third, the symmetry-preserving many-body gapping interactions considered in section~\ref{sec:interaction} have a ground state that exhibits long-range entanglement. This entails degenerate ground states when the system is compactified on a closed three dimensional manifold, and fractional quasi-particle and quasi-string excitations or defects. These topological order properties were not elaborated in our current work but will be crucial in understanding the topological phase~\cite{SirotaRazaTeoappearsoon} as well as the future designs of detection and observation. It would also be interesting to explore possible relationships between the coupled wire construction and alternative exotic states in three dimensions, such as the Haah's code~\cite{Haah11,Haah13}.

Fourth, the many-body inter-wire backscatterings proposed in section~\ref{sec:interactionmodels} were based on a fractionalization scheme described in \ref{sec:gluing} that decomposes a chiral Dirac channel with $(c,\nu)=(1,1)$ into a decoupled pair of Pfaffian ones each with $(c,\nu)=(1/2,1/2)$. In theory, there are more exotic alternative partitions. For instance, if a Dirac channel can be split into three equal parts instead of two, an alternative coupled wire model that put Dirac channels on a honeycomb vortex lattice could be constructed by backscattering these fractionalized channels between adjacent pairs of wires. Such higher order decompositions may already be available as conformal embeddings in the conformal field theory context. For example, the affine $SU(2)$ Kac-Moody theory at level $k=16$ has the central charge $c=8/3$, and its variation may serve as the basis of a "ternionic" model.

%Continuum model
%Material realization?
%Melting
%Defect modes
%Linear response theory