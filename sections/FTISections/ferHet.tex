We begin with a slab that has opposite time reversal breaking ferromagnetic surfaces. In the ferromagnetic surfaces, in addition to the single-body Dirac mass $m$ for the surface parton, the $\mathbb{Z}_{2n+1}$ gauge sector also shows a time reversal breaking signature. The $\mathbb{Z}_{2n+1}$ gauge theory is only present inside the fractional topological insulator, and when a flux line $\Phi$ terminates at the surface, the time reversal breaking boundary condition confines an electrically neutral surface gauge quasiparticle, denoted by $\zeta^a$, with gauge charge $a$ at the flux-surface junction (see Fig.~\ref{fig1}). This gauge flux-charge composite, referred to as a dyon $\delta=\Phi\times\zeta^a$, carries fractional spin $h_\delta=a/(2n+1)$ because a $2\pi$-rotation about the normal axis braids $a$ gauge charges around $\Phi$ and results in the monodromy quantum phase of $e^{2\pi ia/(2n+1)}$. Time reversal conjugates all quantum phases so, $a\not\equiv0$ modulo $2n+1$ breaks time reversal.

The one-dimensional interface between two time reversal conjugate ferromagnetic surface domains hosts a fractional chiral channel. For example, the interface between two ferromagnetic domains with opposite ferromagnetic orientations on the surface of a conventional topological insulator bounds a chiral Dirac channel~\cite{TeoKane,FuKanechargetransport09,QiWittenZhang13}, where electrons propagate only in the forward direction. Alternatively, a topological insulator slab with opposite time reversal breaking ferromagnetic surfaces is topologically identical to a quasi-$(2+1)$-D Chern insulator~\cite{Haldane1988,liu2016quantum} and supports a chiral Dirac edge mode. Similarly, in the fractional topological insulator case, the low-energy content of the fractional chiral channel between a pair of time reversal conjugate ferromagnetic surface domains can be inferred by the edge mode of a fractional topological insulator slab with time reversal breaking ferromagnetic surfaces that is topologically identical to a quasi-$(2+1)$-D fractional Chern insulator~\cite{RegnaultBernevigfractionchern,NeupertSantosChamonMudry11,TangMeiWen11,ShengGuSunSheng11} or fractional quantum Hall (\hypertarget{FQH}{FQH}) state~\cite{FQHE_Review}. The chiral $(1+1)$-D channel is characterized by two response quantities~\cite{Laughlin_IQHE, Halperin82, Hatsugai93, Schulz00, Volovik92, KaneFisher97, Cappelli01, Kitaev06, Luttinger64} -- the differential electric conductance $\sigma=dI/dV=\nu e^2/h$ that relates the changes of electric current and potential, and the differential thermal conductance $\kappa=dI_T/dT=c(\pi^2k_B^2/3h)T$ that relates the changes of energy current and temperature. In the slab geometry, they are equivalent to the Hall conductance $\sigma=\sigma_{xy}$, $\kappa=\kappa_{xy}$. $\nu=N_e/N_\phi$ is also referred to as the filling fraction of the fractional topological insulator slab and associates the gain of electric charge (in units of $e$) to the addition of a magnetic flux quantum $hc/e$. $c=c_R-c_L$ is the chiral central charge of the conformal field theory (\hypertarget{CFT}{CFT})~\cite{bigyellowbook} that effectively describes the low-energy degrees of freedom of the fractional chiral channel. 

Since the top and bottom surfaces of the fractional topological insulator slab are time reversal conjugate, their parton Dirac masses $m$ and gauge flux-charge ratio $a$ have opposite signs. The anyon content is generated by the partons and gauge dyons. When a gauge flux passes through the entire slab geometry from the bottom to the top surface, it associates with total $2a$ gauge charges at the two surface junctions. We denote this dyon by $\gamma=\Phi\times\zeta^{2a}$, which corresponds to an electrically neutral anyon in the slab with spin $h_\gamma=2a/(2n+1)$. If $a$ is relatively prime with $2n+1$, the primitive dyon generates the chiral Abelian topological field theory $\mathbb{Z}_{2n+1}^{(2a)}$~\cite{MooreSeiberg89,Bondersonthesis}, which consists of the dyons $\gamma^m$, for $m=0,\ldots,2n$, with spins $h_{\gamma^m}=2am^2/(2n+1)$ modulo 1 and fusion rules $\gamma^m\times\gamma^{m'}=\gamma^{m+m'}$, $\gamma^{2n+1}=\gamma^0=1$. In particular, when $a=-1$, $\gamma^n$ now has spin $-2n^2/(2n+1)\equiv n/(2n+1)$ modulo 1, which is identical to that of the fundamental quasiparticle of the $SU(2n+1)$ Chern-Simons theory at level 1~\cite{MooreSeiberg89,Bondersonthesis}. This identifies the Abelian theories $\mathbb{Z}_{2n+1}^{(-2)}\cong\mathbb{Z}_{2n+1}^{(n)}=SU(2n+1)_1$, which has chiral central charge $c_{\mathrm{neutral}}=2n$.

$\mathbb{Z}^{(n)}_{2n+1}=\{{\bf e}^l:l=0,1,\ldots,2n\}$ is the anyon content of the Abelian Chern-Simons $SU(2n+1)_1$ theory with Lagrangian density $\mathcal{L}_{2+1}=\frac{1}{4\pi}\int_{2+1}K_{IJ}\alpha^I\wedge d\alpha^J$, where $\alpha^I$ for $I=1,\ldots,2n$ are $U(1)$ gauge fields, and \begin{align}K_{SU(2n+1)_1}=\left(\begin{array}{*{20}c}2&-1&&&&\\-1&2&-1&&&\\&-1&2&&&\\&&&\ddots&&\\&&&&2&-1\\&&&&-1&2\end{array}\right)\end{align} is the Cartan matrix of $SU(2n+1)_1$. 

The fractional topological insulator slab also supports fractionally charged partons $\psi$, each carrying a gauge charge $g$. The electrically charged sector can be decoupled from the neutral $\mathbb{Z}_{2n+1}^{(2a)}$ sector by combining each parton with a specific number of dyons $\lambda=\psi\times\gamma^{-n^2ug}$, where $ua+v(2n+1)=1$ for some integer $u$, $v$, so that the combination is local (i.e.~braids trivially) with any dyons $\gamma^m$. $\lambda$ has fractional electric charge $q_\lambda=e^\ast$ and spin $h_\lambda=1/2+n^3ug^2/(2n+1)$ modulo 1. The $\langle\mathrm{charge}\rangle$ sector consists of the fractional Abelian quasiparticle products $\lambda^m$, where $\lambda^{2n+1}\sim\psi^{2n+1}\sim\psi_{\mathrm{el}}$ corresponds to the local electronic quasiparticle. In particular, when $a=-1$ and $g=-2$, $h_\lambda=1/2(2n+1)$ and therefore $\lambda$ behaves exactly like the Laughlin quasiparticle of the fractional quantum hall state $U(1)_{(2n+1)/2}$ with filling fraction $\nu=1/(2n+1)$ and chiral central charge $c_{\mathrm{charge}}=1$, which is described by the Chern-Simons Lagrangian $(K/4\pi)\alpha\wedge d\alpha$ with $K=2n+1$.
Combining the neutral and charge sectors, the fractional topological insulator slab with time reversal breaking ferromagnetic surfaces has the decoupled tensor product topological order \begin{align}\mathcal{F}=\langle\mathrm{charge}\rangle\otimes\mathbb{Z}_{2n+1}^{(2a)},\label{FTIFSFS}\end{align} and in the special case when $a=-1$ and $g=-2$, it is identical to the Abelian state $U(1)_{(2n+1)/2}\otimes SU(2n+1)_1$, which has a total central charge $c=2n+1$. In general, the filling fraction and chiral central charge are not definite and are subject to surface reconstruction, 
i.e.~adding electronic Dirac fermions. For example, the fractional topological insulator slab can be combined with a Chern insulators of filling $N$, and this will modify the two response quantities by an equal amount $\nu\to\nu+N$, $c\to c+N$. Restricting to the case when the top and bottom ferromagnetic surfaces are time reversal conjugate and fixing $\theta$, the modification $N$ must be even because the number of additional electronic Dirac fermions on each surface must be even. Hence the rational index $\nu-c$ is a topological information characterizing the fractional topological insulator in addition to the magneto-electric $\theta$ angle. 

%Next we move on to superconducting heterostructures. Parafermion zero modes (\hypertarget{PZM}{PZM}), which are non-Abelian twist defects~\cite{Bombin,KitaevKong12,YouWen,BarkeshliJianQi,Teotwistdefectreview} that generalizes zero energy Majorana bound states~\cite{HasanKane10,QiZhangreview11,Alicea12,Beenakker11,Stanescu_Majorana_review,RMP} and can serve as building blocks of a topological quantum computer~\cite{Kitaev97,OgburnPreskill99,ChetanSimonSternFreedmanDasSarma}, appear at the terminals of superconducting trenches of a fractional Chern insulator~\cite{LindnerBergRefaelStern, ClarkeAliceaKirill, MChen, Vaezi, mongg2}. A quasi-1D trench in the \FTI slab supports counter-propagating low-energy channels, described by the \CFT in \eqref{FTIFSFS}, along the opposite sides. The channels can be gapped and the slab can be glued back together by insulating or pairing backscattering interactions. A \PZM is sandwiched between the insulating and superconducting segments along the trench. It is a twist defect in the sense that when an anyon orbits around the point junction, it changes type according to the parton conjugation $\psi\to\psi^\dagger$ when passing across the parton pair condensate along the superconducting segment. 
%electron backscattering ${\psi_{\mathrm{el}}^R}^\dagger\psi_{\mathrm{el}}^L$ or a pairing ${\psi_{\mathrm{el}}^R}\psi_{\mathrm{el}}^L$. The former insulating interaction condenses $\lambda_R^\dagger\lambda_L$ and $\gamma_R$