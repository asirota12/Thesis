Lastly, we describe the generalized $\mathcal{T}$-Pfaffian surface state that preserves both time reversal and charge $U(1)$ symmetries of the fractional topological insulator. Generalizing the $\mathcal{T}$-Pfaffian symmetric gapped surface state of a conventional toplogical insulator described in Ref.\cite{ChenFidkowskiVishwanath14}, the fractional topological insulator version -- referred here as $\mathcal{T}$-Pfaffian$^\ast$ -- consists of the Abelian surface anyons $\openone_j$ and $\Psi_j$, for $j$ even, and the non-Abelian Ising-like anyons $\Sigma_j$, for $j$ odd. The index $j$ corresponds to the fractional electric charge $q_j=je/4(2n+1)$. The surface anyons satisfy the fusion rules \begin{gather}\openone_j\times\openone_{j'}=\Psi_j\times\Psi_{j'}=\openone_{j+j'},\quad\openone_j\times\Psi_{j'}=\Psi_{j+j'},\nonumber\\\Psi_j\times\Sigma_{j'}=\Sigma_{j+j'},\quad\Sigma_j\times\Sigma_{j'}=\openone_{j+j'}+\Psi_{j+j'},\label{TPffusion}\end{gather} and the spin statistics \begin{gather}h_{\openone_j}=h_{\Psi_j}-\frac{1}{2}=\frac{j^2}{16},\quad h_{\Sigma_j}=\frac{j^2-1}{16}\quad\mbox{modulo 1}\end{gather} so that $\openone_j,\Psi_j$ are bosonic, fermionic or semionic, and $\Sigma_j$ are bosonic or fermionic. The fermion $\Psi_4$ is identical to the super-selection sector of the bulk parton $\psi$, which is local with respect to all surface anyons and can escape from the surface and move into the bulk. Time reversal symmetry acts on the surface anyons the same way it acts on those in the $\mathcal{T}$-Pfaffian state for conventional topological insulator~\cite{ChenFidkowskiVishwanath14,ChoTeoFradkin17}. For example, the parton combinations $\psi^{2j+1}=\Psi_{8j+4}$ (and $\psi^{2j}=\openone_{8j}$) are Kramers doublet fermions (respectively Kramers singlet bosons), while $\Psi_{8j}$ ($\openone_{8j+4}$) are Kramers singlet fermions (respectively Kramers doublet bosons). Moreover, for identical reasons as in the conventional topological insulator case, the $\mathcal{T}$-Pfaffian state is anomalous and can only be supported holographically on the surface of a topological bulk. For instance, the bosonic topological order of the $\mathcal{T}$-Pfaffian state after gauging fermion parity would necessarily have a non-trivial chiral central charge which would 
violate time reversal symmetry. We notice in passing that there are alternative surface topological order that generalize those in Refs.\cite{WangPotterSenthilgapTI13,MetlitskiKaneFisher13b}. However we will only focus on the generalized $\mathcal{T}$-Pfaffian state in this paper.
%We refer these properties to Ref.\cite{ChenFidkowskiVishwanath14}. 
%possible alternative of TPf

The fractional topological insulator slab with a time reversal symmetric generalized $\mathcal{T}$-Pfaffian top surface and a time reversal breaking bottom ferromagnetic surface carries a novel quasi-$(2+1)$-D topological order. Its topological content consists of the fractional partons coupled with the $\mathbb{Z}_{2n+1}$ gauge theory in the bulk and the generalized $\mathcal{T}$-Pfaffian surface state (see Fig.~\ref{fig1}). All surface anyons are confined to the time reversal symmetric surface except the parton combinations $\psi^{2j+1}=\Psi_{8j+4}$ and $\psi^{2j}=\openone_{8j}$. Moreover, the time reversal breaking boundary condition confines a gauge quasiparticle $\zeta^a$ per gauge flux $\Phi$ ending on the ferromagnetic surface. On the other hand, there is no gauge charge associated with a gauge flux ending on the generalized $\mathcal{T}$-Pfaffian surface because of time reversal symmetry. Thus a gauge flux passing through the entire slab corresponds to the dyon $\delta=\Phi\times\zeta^a$ with spin $h_\delta=a/(2n+1)$ modulo 1. The generalized $\mathcal{T}$-Pfaffian state couples non-trivially to the $\mathbb{Z}_{2n+1}$ gauge theory as the parton $\psi=\Psi_4$ carries a gauge charge $g$. The general surface anyons $X_j$, for $X=\openone,\Psi,\Sigma$, must carry the gauge charge $z(j)\equiv n^2gj$ modulo $2n+1$ and associate to the monodromy quantum phase $e^{2\pi iz(j)/(2n+1)}$ when orbiting around the dyon $\delta$. For instance, as $2n\equiv-1$ modulo $2n+1$, $z(4j)\equiv gj$ counts the gauge charge of the parton combination $\psi^j$.

The topological order of this fractional topological insulator slab is therefore generated by combinations of the generalized $\mathcal{T}$-Pfaffian anyons and the dyon $\delta$. We denote the composite anyon by \begin{align}
\tilde{X}_{j,z}=X_j\otimes\delta^{z+n^3ugj},\label{Zfanyon}
\end{align} where $X=\openone,\Psi$ for $j$ even or $\Sigma$ for $j$ odd, $z=0,\ldots,2n$ modulo $2n+1$, and $ua+v(2n+1)=1$. They satisfy the fusion rules \begin{gather}\tilde\openone_{j,z}\times\tilde\openone_{j',z'}=\tilde\Psi_{j,z}\times\tilde\Psi_{j',z'}=\tilde\openone_{j+j',z+z'},\nonumber\\\tilde\openone_{j,z}\times\tilde\Psi_{j',z'}=\tilde\Psi_{j+j',z+z'},\quad\tilde\Psi_{j,z}\times\tilde\Sigma_{j',z'}=\tilde\Sigma_{j+j',z+z'},\nonumber\\\tilde\Sigma_{j,z}\times\tilde\Sigma_{j',z'}=\tilde\openone_{j+j',z+z'}+\tilde\Psi_{j+j',z+z'}.\label{Zffusion}\end{gather} They follow the spin statistics \begin{align}h(\tilde\openone_{j,z})&=h(\tilde\Psi_{j,z})-\frac{1}{2}=h(\tilde\Sigma_{j,z})+\frac{1}{16}\nonumber\\&=\frac{j^2}{16}+\frac{az^2-n^6ug^2j^2}{2n+1}\quad\mbox{modulo 1}.\end{align} The $j,z$ indices in \eqref{Zfanyon} are defined in a way so that $\tilde{X}_{j,0}$ are local with respect to the dyons $\delta^z=\tilde\openone_{0,z}$ and decoupled from the dyon sector $\mathbb{Z}_{2n+1}^{(a)}$. The generalized $\mathcal{T}$-Pfaffian surface anyons belong to the subset $X_j=\tilde{X}_{j,-n^3ugj}$, which is a maximal sub-category that admits a time reversal symmetry. The electronic quasiparticle belongs to the super-selection sector $\psi_{\mathrm{el}}=\tilde\Psi_{4(2n+1),0}$, which is local with respect to all anyons. If one gauges fermion parity and includes anyons that associate $-1$ monodromy phase with $\psi_{\mathrm{el}}$, i.e.~if one includes $\tilde\openone_{j,z},\tilde\Psi_{j,z}$ for $j$ odd and $\tilde\Sigma_{j,z}$ for $j$ even, the $\langle\overline{\mathrm{Ising}}\rangle$ sector generated by $1=\tilde\openone_{0,0}$, $f=\tilde\Psi_{0,0}$, $\sigma=\tilde\Sigma_{0,0}$ is local with and decoupled from the $\langle\mathrm{charge}\rangle_{\mathrm{Pf}^\ast}$ sector generated by $\tilde\openone_{j,0}$. The topological order of the fractional topological insulator slab thus takes the decoupled tensor product form after gauging fermion parity \begin{align}\mathrm{Pf}^\ast=\langle\mathrm{charge}\rangle_{\mathrm{Pf}^\ast}\otimes\langle\overline{\mathrm{Ising}}\rangle\otimes\mathbb{Z}_{2n+1}^{(a)}.\label{ZfTO}\end{align} Gauging fermion parity is not the focus of this paper. Nevertheless, we notice in passing that there are inequivalent ways of fermion parity gauging, and in order for the $\mathrm{Pf}^\ast$ theory to have the appropriate central charge, \eqref{ZfTO} needs to be modified by a neutral Abelian $SO(2n)_1$ sector~\cite{ChoTeoFradkin17}. However, the tensor product \eqref{ZfTO} is sufficient and correct to describe the fermionic topological order of the fractional topological insulator slab (with global ungauged fermion parity) by restricting to super-selection sectors $\tilde{X}_{j,z}$ that are local with respect to the electronic quasiparticle $\psi_{\mathrm{el}}$. We refer to this fermionic topological order as a generalized Pfaffian state.
